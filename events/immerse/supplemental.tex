\documentclass{amsart}

%\usepackage{doublespace}
\usepackage{amsmath, amsfonts, amssymb, amsthm, amscd}

%\usepackage[all]{xy}
\usepackage{enumerate}
\usepackage{graphicx}
%\usepackage{pdfsync}

\usepackage{vmargin}
\setpapersize{USletter}
\setmargrb{1in}{1in}{1in}{1in}


\setlength{\parindent}{0in}
\setlength{\parskip}{2mm}


\theoremstyle{plain}
\newtheorem{theorem}{Theorem}[section]
\newtheorem{lemma}[theorem]{Lemma}
\newtheorem{corollary}[theorem]{Corollary}
\newtheorem{proposition}[theorem]{Proposition}
\newtheorem{conjecture}[theorem]{Conjecture}
\newtheorem{principle}[theorem]{Principle}
\newtheorem{claim}[theorem]{Claim}

\theoremstyle{definition}
\newtheorem{definition}{Definition}
\newtheorem{example}[theorem]{Example}
\newtheorem{answer}[theorem]{Answer}
\newtheorem{remark}[theorem]{Remark}


\newtheorem{exercise}[theorem]{Exercise}
\newtheorem{chexercise}[theorem]{Challenge Exercise}
\newtheorem{paperexercise}[theorem]{Paper Exercise}



\theoremstyle{definition}
%\newtheorem{Definition}[Theorem][section]

%\newtheorem{example}{Example}[section]
%\newtheorem{definition}[theorem]{Definition}

%\input xy
%\xyoption{all}



\newcommand{\defining}[1]{\textbf{#1}}


%\newcommand{\sse}{\subseteq}
%\newcommand{\ssne}{\subsetneq}
\renewcommand{\colon}[3]{\ensuremath{\left(#1:_{#2} #3\right)}}
%\newcommand{\rad}[1]{\ensuremath{\operatorname{rad}\left(#1\right)}}
\DeclareMathOperator{\radname}{rad}
\newcommand{\rad}[1]{\radname(#1)}

\newcommand{\m}{\mathfrak{m}}

\newcommand{\N}{\mathbb{N}}
\newcommand{\Z}{\mathbb{Z}}
\newcommand{\Q}{\mathbb{Q}}
\newcommand{\R}{\mathbb{R}}


\DeclareMathOperator{\interior}{int}
\newcommand{\boundary}{\partial}

\newcommand{\tropicalsemiring}{\textbf{T}}
\newcommand{\troplus}{\oplus}
\newcommand{\tromult}{\odot}

\DeclareMathOperator{\aff}{aff}
\DeclareMathOperator{\conv}{conv}
\DeclareMathOperator{\relint}{relint}

%%%%%%%%%%%%%%%%%%%%%%%%%%%%%%%%%%%%%%%%%%%%%%%%%%%%%%%%%%%%%%%%%%%%%%%%%%%%
\begin{document}

\centerline{\large{\bf IMMERSE 2007}}
\vskip.25in

\centerline{\large{Algebra Exercises}}
\vskip.25in

\setcounter{page}{12}
\setcounter{section}{5}
\section{Supplemental Exercises}



\begin{exercise}
This exercise goes back to review the definition of a ring, and explore an object that is ``almost a ring''.
It should be significantly more elementary than most of the other exercises in the whole course.

The \defining{tropical semiring} $\tropicalsemiring$, also called the \defining{min-plus algebra}, is defined as follows.
As a set, $\tropicalsemiring = \R \cup \{\infty\}$.
There are two operations, denoted $\troplus$ (tropical addition) and $\tromult$ (tropical multiplication),
and defined by the following formulas:
\[ a \troplus b = \min\{ a , b \} , \qquad a \tromult b = a + b \text{ (ordinary addition)} \]
Show the following:
\begin{enumerate}[a.]
\item Find $3 \troplus 7$ and $3 \tromult 7$.
\item Show $\troplus$ and $\tromult$ are associative.
\item Show $\troplus$ and $\tromult$ are commutative.
\item Show $\tromult$ distributes over $\troplus$.
\item Is there an identity element for $\troplus$? Are there tropical additive inverses? Explain.
\item Is there an identity element for $\tromult$? Are there tropical multiplicative inverses? Explain.
\item In $\R^2$, graph the tropical linear polynomial $y = (3 \tromult x) \troplus 7$.
\item In $\R^2$, graph the tropical quadratic polynomial
$y = (3 \tromult x \tromult x) \troplus (7 \tromult x) \troplus 12$.
The expression $x \tromult x$ can be abbreviated $x^{\tromult 2}$ or simply $x^2$.
\item Show the ``freshman's dream'' holds for all powers in tropical geometry:
$(x \troplus y)^{\tromult n} = x^{\tromult n} \troplus y^{\tromult n}$.
\item What is the ``characteristic'' of $\tropicalsemiring$?
\end{enumerate}
\end{exercise}


%\begin{exercise}
%Another ``almost ring''.
%Let $G$ be a group, written additively, but not necessarily abelian.
%Let $N(G)$ be the set of functions from $G$ (arbitrary functions, not just homomorphisms),
%with the two operations $+$ and $\circ$ defined as follows.
%The addition $+$ is defined by pointwise addition:
%$(f+g)(x) = f(x) + g(x)$.
%The composition $\circ$ is $(f \circ g)(x) = f(g(x))$.
%This structure is called the \defining{near-ring of functions on $G$}.
%\begin{enumerate}[a.]
%\item Show $+$ and $\circ$ are associative.
%\item Show there are identity and inverses for $+$,
%and an identity for $\circ$ but not necessarily inverses.
%\item Show $\circ$ is right-distributive over $+$,
%that is, $(f+g)\circ h = (f\circ h) + (g\circ h)$.
%\item A structure $(N,+,\circ)$ satisfying the above properties
%is called a \defining{near-ring}.
%Show $(\R[x], + , \circ)$ is a near-ring, with $+$ commutative.
%\item An element of a near-ring is called distributive
%if it distributes on the left (as well as the right).
%In $N(G)$, an element is distributive if and only if
%it is a homomorphism on $G$.
%\end{enumerate}
%\end{exercise}








\begin{exercise}
Determine all subsets $S$ of $\R^1$ such that
both $S$ and its complement are convex;  do the same for $\R^2$ and $\R^3$.
(For simplicity, you may want to just take $S$ to be closed and its complement open.)
\end{exercise}











\begin{definition}
Let $R$ be a commutative ring with identity and let $I$ and $J$ be ideals of $R$.
The \defining{quotient ideal of $I$ and $J$}, or \defining{colon ideal}, is
\[
\colon{I}{R}{J} = \left\{ r \in R \mid r J \subseteq I \right\}.
\]
It is often written simply $\colon{I}{}{J}$ if the ring $R$ is clear from the context.
\end{definition}

\begin{exercise}\label{exer:colon}
Let $I, J, K$ be ideals of $R$
and let $\left\{ I_{\lambda} \right\}_{\lambda \in \Lambda}$ be a collection of ideals
of $R$. Prove the following statements:
\begin{enumerate}[a.]
\item \colon{I}{}{J} is an ideal of $R$.

\item $\colon{I}{}{J} = R$ if and only if $J \subseteq I$.

\item $I \subseteq \colon{I}{}{J}$.

\item $\colon{I}{}{J} J \subseteq I$.

\item $\colon{\colon{I}{}{J}}{}{K} = \colon{I}{}{JK} =
\colon{\colon{I}{}{K}}{}{J}$.

\item $\colon{\bigcap_{\lambda \in \Lambda} I_{\lambda}}{}{J} = \bigcap_{\lambda \in \Lambda}
\colon{I_{\lambda}}{}{J}$.

\item $\colon{J}{}{\sum_{\lambda \in \Lambda} I_{\lambda}} = \bigcap_{\lambda \in \Lambda}
\colon{J}{}{I_{\lambda}}$. [HINT: Use the fact that $\sum I_{\lambda}$ is the ideal generated by
$\bigcup I_{\lambda}$, which you proved in Exercise~1.11 and Exercise~1.12.]
\end{enumerate}
\end{exercise}


\begin{exercise}
Let $I$ be a radical ideal and $J$ an arbitrary ideal.
Prove that $\colon{I}{}{J}$ is radical.
\end{exercise}

\begin{exercise} % [HS 1.4]
Let $I$ be an integrally closed ideal and $J$ an arbitrary ideal.
Prove that $\colon{I}{}{J}$ is integrally closed.
\end{exercise}

\begin{exercise}
Let $I$ and $J$ be monomial ideals.
Show that $\colon{I}{}{J}$ is a monomial ideal.
%(Do Exercise~\ref{exer:colon} first.)
Also, show that if $I$ and $J$ are homogeneous ideals, so is $\colon{I}{}{J}$.
\end{exercise}

\begin{exercise}
Let $I$ and $J$ be ideals in $R$.
Show the following.
\begin{enumerate}[a.]
\item We have $\colon{ \colon{I}{}{J^a} }{}{J} = \colon{I}{}{J^{a+1}}$.
\item If $a \leq b$ then $\colon{I}{}{J^a} \subseteq \colon{I}{}{J^b}$.
By the previous part, if we keep coloning by $J$, the ideal just keeps growing.
\item The \defining{saturation of $I$ with respect to $J$},
\[ \colon{I}{}{J^{\infty}} = \bigcup_{a \geq 0} \colon{I}{}{J^a} , \]
is an ideal containing $I$, and such that
$\colon{ \colon{I}{}{J^{\infty}} } {} {J^a} = \colon{I}{}{J^{\infty}}$ for all $a$.
(This justifies the name ``saturated'': putting more $J$s doesn't change it any more.)
\item If $I$ and $J$ are monomial, so is $\colon{I}{}{J^{\infty}}$.
\item If $I$ and $J$ are homogeneous, so is $\colon{I}{}{J^{\infty}}$.
\item If $I$ is a homogeneous ideal, we define the \defining{saturation of $I$}
to be $I^{\text{sat}} = \colon{I}{}{\m^{\infty}}$.
Show this is a homogeneous ideal containing $I$ and that $(I^{\text{sat}})^{\text{sat}} = I^{\text{sat}}$.
\item If $I \subset J$ are homogeneous then $I^{\text{sat}} \subset J^{\text{sat}}$.
\item For $I$ homogeneous we have $I \subset I^{\text{sat}} \subset \rad{I}$.
Show a radical homogeneous ideal is saturated.
\item For $I$ homogeneous, $I^{\text{sat}}$ is the unique smallest saturated
ideal containing $I$. (A homogeneous ideal is saturated if it equals its saturation.
Here, ``smallest'' means $I^{\text{sat}}$ is contained in any saturated ideal containing $I$.)
\item Let $I$ be homogeneous and $d \geq 0$.
The \defining{degree $d$ piece of $I$}, denoted $I_d$, is the set of homogeneous forms
of degree $d$ in $I$ (together with $0$).
Show that each $I_d$ is a vector space over $k$, contained in $\m_d$.
Show that $\m_d$ is finite dimensional, and hence so is $I_d$.
\item For $I$ homogeneous, $I_d = (I^{\text{sat}})_d$ for $d \gg 0$.
[HINT: Use the Noetherian property. It says $I^{\text{sat}}$ is finitely generated.
So there is a maximum degree $a$ of the generators, and a $b$
such that all the generators multiply $\m^b$ into $I$
(taking the maximum of the $b$'s of the various generators).
Show that $I_d = (I^{\text{sat}})_d$ for $d \geq a+b$.]
\item For $I$ and $J$ homogeneous, $I^{\text{sat}} = J^{\text{sat}}$
if and only if $I_d = J_d$ for $d \gg 0$.
\end{enumerate}
\end{exercise}


\begin{exercise}
Let $I_1 \subset I_2 \subset \dots$ be an ascending chain of ideals
and $J = \bigcup I_i$ be their union.
Show the following.
\begin{enumerate}[a.]
\item If each $I_i$ is prime, so is $J$.
\item If each $I_i$ is radical, so is $J$.
\item If each $I_i$ is homogeneous (in a polynomial ring), so is $J$.
Ditto, monomial. Ditto, homogeneous and saturated.
\item (Harder): If each $I_i$ is integrally closed, so is $J$.
\end{enumerate}
\end{exercise}


\begin{exercise}
Let $A$ be any commutative ring with $1$ (not necessarily Noetherian)
and let $R = A[x_1,\dots,x_n]$.
Let $I \subset R$ be an ideal generated by a set of monomials.
(Here, a monomial is a polynomial with one term and coefficient $1$.)
Then show $I$ is generated by a finite set of monomials.
[HINT: Imitate the proof given in class that the lexicographic order
is a well-ordering.]

Thus, even though $R$ is not Noetherian if $A$ is not, at least all
ideals generated by monomials are actually finitely generated by monomials.
\end{exercise}


\begin{exercise}
Prove the statement in H\"ubl's Remark~4, (ii): that $K = F + \R_{\geq 0}^{d+1}$.
\end{exercise}



\begin{chexercise}
Let $I \subset R = k[x_1,\dots,x_n]$ be a monomial ideal and $\m = (x_1,\dots,x_n)$.
Prove or disprove the following.
\begin{enumerate}[a.]
\item If $I$ satisfies \textbf{NN}, then so does $I^2$. (What about higher powers?)
\item If $I$ satisfies \textbf{NN}, then so does $\m I$.
\item If $I^2$ satisfies \textbf{NN}, then so does $I$.
\item If $\m I$ satisfies \textbf{NN}, then so does $I$.
\end{enumerate}
If possible, salvage those which are false---that is, change the statement so it becomes true.
For example, strengthen the hypothesis or weaken the conclusion.
(I have no idea if these are true or false; trivial or impossible.)
\end{chexercise}

\begin{chexercise}
For those of you who like polyhedra:
Define property \textbf{NN} for polyhedra.
You may want to restrict to bounded polyhedra (i.e., polytopes),
or lattice polytopes (i.e., convex hull of points in $\Z^n$), or polyhedra $P$
such that $P + v \subseteq P$ if and only if $v \in \R_{\geq 0}^n$,
or possibly other reasonable restrictions.
Once you have a good definition of property \textbf{NN},
rewrite H\"ubl's paper, proving all the analogous theorems for polyhedra.
Write and publish it collaboratively with your IMMERSE classmates
and/or teachers.
\end{chexercise}

\begin{chexercise}
For those of you who like algebra:
Study $I^{p/q} = \{ r : r^q \in I^p \}$
and $\tilde{I}^{p/q} = \bigcup \{ I^{r/s} : \frac{r}{s} = \frac{p}{q} \}$.
Relate them to the better-known ideals
$\overline{I}^{p/q} = \{ r : r^q \in \overline{I^p} \}$.
Write and publish your results collaboratively with your IMMERSE classmates
and/or teachers.
(Warning: Results are not guaranteed. This may lead nowhere.)
\end{chexercise}



\begin{chexercise}
Generalize H\"ubl's Corollary~6 to higher dimensions, that is, more than $2$ variables.

Suggestion: In $\R^2$, a line $ax+by=c$ with $a,b \in \Z$ and $\gcd(a,b)=1$
has slope equal to an integer or the reciprocal of an integer if and only if $a=1$ or $b=1$.
Perhaps rephrasing Corollary~6 using this idea would allow for a nice generalization.
\end{chexercise}


\end{document}
