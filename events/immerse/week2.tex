\documentclass{amsart}

%\usepackage{doublespace}
\usepackage{amsmath, amsfonts, amssymb, amsthm, amscd}

%\usepackage[all]{xy}
\usepackage{enumerate}
\usepackage{graphicx}
%\usepackage{pdfsync}

\usepackage{vmargin}
\setpapersize{USletter}
\setmargrb{0.8in}{1in}{1in}{0.8in}


\setlength{\parindent}{0in}
\setlength{\parskip}{2mm}


\theoremstyle{plain}
\newtheorem{theorem}{Theorem}[section]
\newtheorem{lemma}[theorem]{Lemma}
\newtheorem{corollary}[theorem]{Corollary}
\newtheorem{proposition}[theorem]{Proposition}
\newtheorem{conjecture}[theorem]{Conjecture}
\newtheorem{principle}[theorem]{Principle}
\newtheorem{claim}[theorem]{Claim}

\theoremstyle{definition}
\newtheorem{definition}{Definition}[section]
\newtheorem{example}[theorem]{Example}
\newtheorem{answer}[theorem]{Answer}
\newtheorem{remark}[theorem]{Remark}


\newtheorem{exercise}[theorem]{Exercise}
\newtheorem{paperexercise}[theorem]{Paper Exercise}



\theoremstyle{definition}
%\newtheorem{Definition}[Theorem][section]

%\newtheorem{example}{Example}[section]
%\newtheorem{definition}[theorem]{Definition}

%\input xy
%\xyoption{all}



\newcommand{\defining}[1]{\textbf{#1}}


\newcommand{\sse}{\subseteq}
%\newcommand{\ssne}{\subsetneq}
%\renewcommand{\colon}[3]{\ensuremath{\left(#1:_{#2} #3\right)}}
%\newcommand{\rad}[1]{\ensuremath{\operatorname{rad}\left(#1\right)}}
\DeclareMathOperator{\radname}{rad}
\newcommand{\rad}[1]{\radname(#1)}

\newcommand{\m}{\mathfrak{m}}

\newcommand{\N}{\mathbb{N}}
\newcommand{\Z}{\mathbb{Z}}
\newcommand{\Q}{\mathbb{Q}}
\newcommand{\R}{\mathbb{R}}


\DeclareMathOperator{\interior}{int}
\newcommand{\boundary}{\partial}

%%%%%%%%%%%%%%%%%%%%%%%%%%%%%%%%%%%%%%%%%%%%%%%%%%%%%%%%%%%%%%%%%%%%%%%%%%%%
\begin{document}

\setcounter{page}{3}

\centerline{\large{\bf IMMERSE 2007}}
\vskip.25in

\centerline{\large{Algebra Exercises}}
\vskip.25in

\setcounter{section}{1}
\section{Week 2}


Unless otherwise noted, the letters $R$ and $A$ denote commutative rings with unity,
the letters $I$ and $J$ denote ideals, the letter $k$ denotes a field,
and letters like $X_i$ denote variables.


\begin{definition}
Let $\sigma: \R^n \to \R$ be a linear function and let $s$ be a real number.
The \defining{closed half space} in $\R^n$ corresponding to $\sigma$ and $s$
is the set  $\{ v \in \R^n: \sigma(v) \geq s \}$.
The \defining{open half space} in $\R^n$ corresponding to $\sigma$ and $s$
is the set $\{ v \in \R^n: \sigma(v) > s \}$.
\end{definition}

\begin{exercise}
Let $\sigma: \R^n \to \R$ be a linear function and let $s$ be a real number.
Let $\Sigma$ be the corresponding closed half space
and let $\Sigma^\circ$ be the corresponding open half space.
\begin{enumerate}[a.]
\item Show that $\Sigma$ and $\Sigma^\circ$ are convex.
\item Show that $\Sigma$ is the topological closure of $\Sigma^\circ$
and $\Sigma^\circ$ is the topological interior of $\Sigma$.
\end{enumerate}
\end{exercise}

\begin{paperexercise} \label{paperex:lemma1example}
For the the monomial ideal $I = (X_0^4, X_0^3X_1^3, X_1^4)$ (as on page 3772 of the H\"ubl paper),
determine $C$, $K$, $\sigma_i$, $s_i$, and $\Sigma_i$, as in Lemma 1.
\end{paperexercise}




\begin{exercise}
For each of the sets in Exercise~1.15 %was: \ref{Exercise:IsAnIdeal?} from weekly1.tex
that is an ideal, find a finite generating set.
Prove that the set actually generates the ideal.
\end{exercise}


\begin{exercise}
Prove that an ideal $I$ of $R$ is maximal if and only if $R/I$ is a field.
\end{exercise}

\begin{exercise}
Prove that $R$ is a field if and only
if the only ideals of $R$ are $(0_{R})$ and $R$.
\end{exercise}

\begin{exercise}
Show that if $R$ is a commutative ring, then $R[x]$ is never a field.
%HINT: if $x^{-1}$ exists, what is its degree?
\end{exercise}

\begin{exercise} 

\begin{enumerate}[a.]

\item Say $R$ is a domain. Show that if a polynomial in $R[x]$ is a unit, then it is a nonzero constant.

\item Say $R=k$ is a field. Then the converse is true: if a polynomial in $k[x]$ is a nonzero constant,
it is a unit.

\item Show that $(2x+1)^2 = 1$ in $(\Z/4)[x]$.
Conclude that the hypothesis in part (a) that $R$ be a domain cannot be removed in general.
\end{enumerate}
\end{exercise}


\begin{definition}
Let $A$ be a domain. An element $f \in A$ is called \defining{irreducible}
if $f \neq 0$, $f$ is not a unit, and for any factorization $f = gh$, either $g$ or $h$ is a unit.
\end{definition}

\begin{exercise}
Let $f \in k[x]$ be a non-constant polynomial in one variable.
Prove that the following are equivalent:
(1) $k[x] / \langle f \rangle$ is a field.
(2) $k[x] / \langle f \rangle$ is an integral domain.
(3) $f$ is irreducible.
\end{exercise}


\begin{exercise}
Let $f = Y^2 - X^3 \in k[X,Y]$.
Show $f$ is irreducible.
Is $k[X,Y] / \langle f \rangle$ a field, an integral domain, or neither?
\end{exercise}


\begin{definition}
Let $R$ be a commutative ring with identity and let $I$ be an ideal of $R$.
The \defining{radical} of $I$ is the following set:
$\rad{I} = \left\{ x \in R \mid x^{n} \in I \text{ for some } n > 0  \right\}$.
Other common notations include $\sqrt{I}$ and $\text{r}(I)$.
\end{definition}


\begin{exercise}
Let $R = \mathbb{Z}$.
Find the radicals of the ideals: $(12)$, $(14)$, $(16)$, $(18)$.
\end{exercise}


\begin{exercise}
%Let $R$ be a commutative ring with identity, let
%Let $n$ be a positive integer, and let
%$I$, $J$, and
%$I_{1}, I_{2}, \ldots, I_{n}$ be ideals of $R$.
Prove the following statements:
\begin{enumerate}[a.]
\item $\rad{I}$ is an ideal of $R$.
[HINT: To show closure under addition, use the binomial theorem.]

\item $I \sse \rad{I}$.

\item $\rad{I} = \rad{\rad{I}}$.

\item $\rad{I J} = \rad{I \cap J} = \rad{I} \cap \rad{J}$.

\item $\rad{I} = R$ if and only if $I = R$. [HINT: Use $1 \in R$.]

\item $\rad{I + J} = \rad{\rad{I} + \rad{J}}$.

\item $I \sse J$ implies $\rad{I} \sse \rad{J}$.

\item Suppose $I = (f_{1}, \ldots, f_{m})$.  Then $\rad{I} \sse
\rad{J}$ if and only if for each $i = 1, 2, \ldots, m$ there exists a positive
integer $n_{i}$ such that $f_{i}^{n_{i}} \in J$.

\item Suppose $I = (f_{1}, \ldots, f_{s})$ and $J = (g_{1}, \ldots, g_{t})$.
Then $\rad{I} = \rad{J}$ if and only if for each $i = 1, 2, \ldots, s$ there exists a
positive integer $n_{i}$ such that $f_{i}^{n_{i}} \in J$, and for each $j = 1, 2,
\ldots, t$ there exists a positive integer $m_{j}$ such that $g_{j}^{m_{j}} \in I$.

\item Suppose $I \sse J$ and that $J = (g_{1}, \ldots, g_{t})$.
Then $\rad{I} = \rad{J}$ if and only if for each $j = 1, 2, \ldots, t$ there exists an integer
$m_{j}$ such that $g_{j}^{m_{j}} \in I$.
%
%\item \rad{I} is the intersection of all the prime ideals that contain $I$.
\end{enumerate}
\end{exercise}








\begin{exercise} \label{exer:product ideal} % [M.~Artin]
Let $I$ and $J$ be ideals of $R$.
\begin{enumerate}[a.]
\item Prove that $I \cap J$ is an ideal.

\item Show by example that the set of products $\{xy : x \in I, y \in J \}$ need not be an ideal.

\item But show that the set of finite sums
of products of elements of $I$ and $J$ is an ideal.
That is, show the set
$\{ \sum_{k=1}^{n} x_k y_k : \text{$n \geq 0$, and for each $k$, $x_k \in I, y_k \in J$} \}$
is an ideal.

It is called the \defining{product ideal of $I$ and $J$} and denoted $IJ$.

\item Prove that $IJ \subseteq I \cap J$.

\item Show by example that $IJ$ and $I \cap J$ need not be equal.
\end{enumerate}
\end{exercise}



\begin{exercise} \label{exer:generating sets}
Say $I$ is generated by the set $S$ and $J$ is generated by the set $T$.
True or false:
\begin{enumerate}[a.]
\item $I \cap J$ is generated by $S \cap T$.
\item $I J$ is generated by $\{ s t : s \in S, t \in T \}$.
\item $I^2 = II$ is generated by $\{ s^2 : s \in S \}$.
\end{enumerate}
(For ``true'', give a proof; for ``false'', give a counterexample.)
\end{exercise}



\begin{paperexercise}
Let $I$ be the same ideal as in exercise~\ref{paperex:lemma1example}.
Find a generating set for $I^2$.
Repeat exercise~\ref{paperex:lemma1example} for $I^2$.
\end{paperexercise}



%\begin{exercise} [RS]
%Let $I$ be a monomial ideal in $R=A[x_1,\ldots,x_d]$ where $d\geq 2$.
%Let $f$ be a monomial in $R$.
%\begin{enumerate}[a.]
%\item If $x_i\mid f$, then $\rad{fI}\subseteq \rad{(x_i)R}$.
%\item If $f\neq 1_A$, then $\rad{fI}\subsetneq\rad{X}$.
%\end{enumerate}
%\end{exercise}

Let $X_1, \dots, X_d$ be variables.
For any vector of non-negative integers $\alpha \in \R^d$
we may write $X^\alpha$ to denote $X_1^{\alpha_1} \cdots X_d^{\alpha_d}$.
Also, we may write $k[X]$ to denote $k[X_1,\dots,X_d]$,
provided it is clear from context whether $X$ denotes a single variable
or the collection of variables $X_1,\dots,X_d$.


\begin{exercise}
Let $X_1, \dots, X_d$ be variables. We use the notation above, so $X$ denotes
the collection of variables.
Let $c_{\alpha}X^{\alpha}$ be a nonzero monomial,
and let $f(X), g(X) \in k[X]$ be polynomials
none of whose terms is divisible by $c_{\alpha}X^{\alpha}$.
Prove that none of the terms of $f(X) - g(X)$
is divisible by $c_{\alpha}X^{\alpha}$. 
\end{exercise}


\begin{exercise} 
Let $R=k[x_1,\ldots,x_d]$ and $\m=(x_1,\ldots,x_d)R$. Show the following.
\begin{enumerate}[a.]
\item If $I$ is a monomial ideal such that $I\neq R$, then $\rad{I}=\m$ if and
only if for each $i=1,\ldots,d$ there exists an integer $n_i>0$ such
that $x_i^{n_i}\in I$.
\item The ideal $\m$ is radical, that is, $\rad{\m} = \m$.
\end{enumerate}
\end{exercise}


\begin{exercise} 
Let $I_1,\ldots,I_n$ be monomial ideals in $R=k[x_1,\ldots,x_d]$ and
set $\m=(x_1,\ldots,x_d)R$.
\begin{enumerate}[a.]
\item Prove that the sum $I_1+\cdots+ I_n$ is a monomial ideal.
\item If $\rad{I_1}=\cdots=\rad{I_n}=\m$, show that $\rad{I_1+\cdots+ I_n}=\m$.
\item Prove or disprove:  If $\rad{I_1+\cdots+ I_n}=\m$, then $\rad{I_1}=\cdots=\rad{I_n}=\m$.
\end{enumerate}
\end{exercise}


\begin{exercise} 
Let $I_1,\ldots,I_n$ be monomial ideals in $R=k[x_1,\ldots,x_d]$ and
set $\m=(x_1,\ldots,x_d)R$.
\begin{enumerate}[a.]
\item Prove that the product $I_1\cdots I_n$ is a monomial ideal.
\item If $\rad{I_1}=\cdots=\rad{I_n}=\m$, show that $\rad{I_1\cdots I_n}=\m$.
\item Prove or disprove:  If $\rad{I_1\cdots I_n}=\m$, then $\rad{I_1}=\cdots=\rad{I_n}=\m$.
\end{enumerate}
\end{exercise}



%\begin{exercise} Let $A$ be a ring and $f = \sum_{|\alpha| \geq 0} a_{\alpha}x^{\alpha} \in A[x_1, \dots, x_n]$.  Prove the following statements:
%
%\begin{enumerate}
%\item $f$ is nilpotent $\Leftrightarrow a_{\alpha}$ is nilpotent for all $\alpha$.
%[HINT: choose a monomial ordering and argue by induction on the number of summands.]
%In particular, $A[x_1, \dots, x_n]$ is reduced if and only if $A$ is reduced.

%\item $f$ is a unit in $A[x_1, \dots, x_n]$ if and only if $A_{0, \dots, 0}$ is a unit in $A$
%and $A_{\alpha}$ are nilpotent for $\alpha \neq 0$.
%[HINT: Remember the geometric series for $1/(a-g)$ and use part (1).]
%In particular, $(A[x_1, \dots, x_n])* = A*$ if and only if $A$ is reduced.

%\item $f$ is a zero divisor in $A[x_1, \dots, x_n]$ if and only if
%there exists some $a \neq 0$ in $A$ such that $af = 0$.
%Give two proofs: one by induction on $n$, the other by using a monomial ordering.
%[HINT: Choose a monomial ordering and $g \in A[x_1, \dots, x_n]$
%with minimal number of terms so that $f \cdot g = 0$, consider the biggest term
%and conclude that $g$ must be monomial.]

%\item $A[x_1, \dots, x_n]$ is an integral domain if and only if
%$\deg(fg) = \deg(f) + \deg(g)$ for all $f, g, \in A[x_1, \dots, x_n]$.
%In particular, $A[x_1, \dots, x_n]$ is an integral domain if and only if $A$ is an integral domain.
%\end{enumerate}
%\end{exercise}



%\begin{definition}
%Let $R$ be a commutative ring with unity.
%The \defining{Jacobson radical} of $R$ is the intersection of all the maximal ideals of $R$.
%The \defining{nilradical} of $R$ is the radical of the zero ideal, that is, $\rad{0}$.
%\end{definition}

%\begin{exercise}
%Let $A$ be a commutative ring with unity.
%In the ring $R=A[x]$, the Jacobson radical is equal to the nilradical.
%\end{exercise}

\begin{definition}
An \defining{ascending chain} of ideals is an infinite sequence
$I_1 \subseteq I_2 \subseteq I_3 \subseteq \cdots$.
An ascending chain is said to \defining{stabilize} if there is some $N$ such that
for $n \geq N$, $I_n = I_N$.
A ring $R$ is called \defining{Noetherian} if every ascending chain of ideals in $R$ stabilizes
(the ``ascending chain condition'').
\end{definition}


\begin{exercise}
Let $I_1 \subseteq I_2 \subseteq \cdots$ be an ascending chain of ideals.
Then $I = \bigcup I_i = I_1 \cup I_2 \cup \cdots$ is an ideal.
\end{exercise}


\begin{exercise}
Let $A$ be a Noetherian ring. Show the following.
\begin{enumerate}[a.]
\item Every field $k$ is Noetherian. [HINT: What ideals are there? What chains are possible?]
\item If $I \subset A$ is an ideal, then $A/I$ is Noetherian.
\item If $I \subset A$ is an ideal, then there is a finite generating set for $I$.
[HINT: Consider the set of all finitely generated ideals contained in $I$.
Use the Noetherian condition plus Zorn's Lemma to show that among all finitely
generated ideals contained in $I$ there is a maximal one.
Now that must be $I$, otherwise adding one more element to the generating set of the ideal
would give a larger ideal, still finitely generated, violating maximality.]
\item Conversely, if every ideal in $R$ is finitely generated, then $R$ is Noetherian.
[HINT: Given a chain, the union of the chain is an ideal and hence finitely generated.
The members of the generating set must lie in some ideals in the chain. Take a maximum.]

Hence: \textit{A ring $R$ is Noetherian if and only if every ideal of $R$
is finitely generated.}
\item (Hilbert's Basis Theorem) The ring $A[X]$ is Noetherian.
[HINT: Given an ideal $I \subset A[X]$, show that the set of leading coefficients of polynomials in $I$
forms an ideal in $A$. By part (c), that ideal is finitely generated.
For each of those generators $a_i$, pick a lowest-degree polynomial $f_i$ with leading coefficient $a_i$.
Show the $f_i$ generate $I$ by considering a lowest-degree member of $I$
outside the ideal generated by the $f_i$ and obtaining a contradiction.]

Hence: \textit{For a field $k$, every ideal in a polynomial ring $k[X_1,\dots,X_d]$ is finitely generated.}
\item The previous part implies that $k[X,Y]$ is Noetherian.
Let $R \subset k[X,Y]$ be the subset of polynomials whose terms are of the form
either $c X^a Y^b$ with $b \leq a \sqrt{2}$ and $c \in k$.
(Note that constant terms, $a=b=0$, meet this condition, so they are allowed.)
For example, $X \in R$ but $Y \notin R$.

Now, show that $R$ is a subring. Then show that $R$ is not Noetherian.


Thus a subring of a Noetherian ring is not necessarily Noetherian.
\end{enumerate}
\end{exercise}



%\begin{exercise}
%Determine whether or not the following polynomials $f$ are contained in the respective ideals~$I$:

%\begin{enumerate}
%\item $f = xy^3 - z^2 + y^5 - z^3, I = \langle -x^3+y, x^2y-z \rangle$ in $\Q[x,y,z]$

%\item $f = x^3z - 2y^2, I = \langle yz + 2z^2, y-z \rangle$ in $\Q[x,y,z]$
%\end{enumerate}
%\end{exercise}





\end{document}
