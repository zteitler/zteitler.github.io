\documentclass{amsart}

%\usepackage{doublespace}
\usepackage{amsmath, amsfonts, amssymb, amsthm, amscd}

\usepackage[all]{xy}
\usepackage{enumerate}
\usepackage{graphicx}
%\usepackage{pdfsync}

\usepackage{vmargin}
\setpapersize{USletter}
\setmargrb{1in}{1in}{1in}{1in}


\setlength{\parindent}{0in}
\setlength{\parskip}{2mm}


\theoremstyle{plain}
\newtheorem{theorem}{Theorem}[section]
\newtheorem{lemma}[theorem]{Lemma}
\newtheorem{corollary}[theorem]{Corollary}
\newtheorem{proposition}[theorem]{Proposition}
\newtheorem{conjecture}[theorem]{Conjecture}
\newtheorem{principle}[theorem]{Principle}
\newtheorem{claim}[theorem]{Claim}

\theoremstyle{definition}
\newtheorem{definition}{Definition}
\newtheorem{example}[theorem]{Example}
\newtheorem{answer}[theorem]{Answer}
\newtheorem{remark}[theorem]{Remark}


\newtheorem{exercise}[theorem]{Exercise}



\theoremstyle{definition}
%\newtheorem{Definition}[Theorem][section]

%\newtheorem{example}{Example}[section]
%\newtheorem{definition}[theorem]{Definition}

\input xy
\xyoption{all}



\newcommand{\defining}[1]{\textbf{#1}}


\newcommand{\sse}{\subseteq}
%\newcommand{\ssne}{\subsetneq}
\renewcommand{\colon}[3]{\ensuremath{\left(#1:_{#2} #3\right)}}
\newcommand{\rad}[1]{\ensuremath{\operatorname{rad}\left(#1\right)}}

\newcommand{\N}{\mathbb{N}}
\newcommand{\Z}{\mathbb{Z}}
\newcommand{\R}{\mathbb{R}}


\DeclareMathOperator{\interior}{int}

%%%%%%%%%%%%%%%%%%%%%%%%%%%%%%%%%%%%%%%%%%%%%%%%%%%%%%%%%%%%%%%%%%%%%%%%%%%%
\begin{document}

\centerline{\large{\bf IMMERSE 2007}}
\vskip.25in

\centerline{\large{Algebra Exercises}}
\vskip.25in

\section{Week 1}

Unless noted otherwise, the letters $R$ and $A$ denote commutative rings with identity.
The letters $I$ and $J$ denote ideals.

\begin{exercise}
Let $R$ be a ring (not necessarily commutative, not necessarily with identity).
For any $x \in R$, $x \cdot 0_R = 0_R \cdot x = 0_R$.
\end{exercise}


\begin{exercise}
Let $R$ be a commutative ring with identity and $I$  an ideal of $R$.
\begin{enumerate}[a.]
\item Show that $0_{R} \in I$.
\item Show that if $a \in I$, then $-a \in I$.
\item Show that if $a, b \in I$, then $a - b \in I$.
\item Show that the set $\{ 0_{R} \}$ is an ideal of $R$.
\item Show that $R$ is an ideal of $R$.
\end{enumerate}
\end{exercise}

\begin{exercise} \label{firstproduct}
Let
%$R$ be a commutative ring with identity.  Let
%$I$ be an ideal of $R$ and let
$r\in R$ and set
$$r I = \left\{ r a \mid a \in I \right\}.$$
Show that $r I$ is an ideal of $R$.
\end{exercise}


\begin{exercise}
%Let $R$ be a commutative ring with identity; as usual, denote the identity by 1.  Let
%$I$ be an ideal of $R$.
Prove that the following statements are equivalent:
\begin{enumerate}[(i)]
\item $I = R$.
\item $1_{R} \in I$.
\item $I$ contains a unit.
\end{enumerate}
\end{exercise}

\begin{exercise}
%Let $R$ be a commutative ring with identity and l
Let $\left\{ I_{\alpha}
\right\}_{\alpha \in \Lambda}$ be a collection of ideals of $R$, where $\Lambda$ is
an index set. (Do not assume that $\Lambda$ is finite or even countable.)  Set $K =
\cap_{\alpha \in \Lambda} I_{\alpha}$.  Prove that $K$ is an ideal of $R$.
(Observe  that $K\sse I_{\alpha}$ for each $\alpha\in \Lambda$.)
\end{exercise}

\begin{exercise}
%Let $R$ be a commutative ring with identity.
\begin{enumerate}[a.]
\item Let $f \in R$.  Prove that $(f)R$ is an ideal of $R$ and that $f\in (f)R$.
%We will call
%$(f)R$ the \emph{ideal generated by $f$.}
\item Let $f_{1}$, \ldots, $f_{s}$ be elements of $R$.  Prove that $(f_{1},
\ldots, f_{s})R$ is an ideal of $R$ and that $f_1,\ldots,f_s\in (f_{1},
\ldots, f_{s})R$.
%We will call $(f_{1}, \ldots, f_{s})R$ the
%\emph{ideal generated by $f_{1}, \ldots, f_{s}$.}
\item Let $S \sse R$.  Prove that $(S)R$ is an ideal of $R$ and that $S\sse (S)R$.
%We will
%call $(S)R$ the \emph{ideal generated by the set $S$.}
\end{enumerate}
\end{exercise}

\begin{exercise}\label{Exercise:IdealContainment}
%Let $R$ be a commutative ring with identity and let $I \sse R$ be an ideal of $R$.
Let $S \sse R$.  Prove that the following statements are equivalent:
\begin{enumerate}[(i)]
\item $S \sse I.$
\item $(S)R \sse I$.
\end{enumerate}
This fact is useful when you want to show that one ideal is contained in another.
\end{exercise}

\begin{exercise}
%Let $R$ be a commutative ring with identity and l
Let $S \sse R$. Let $\left\{
I_{\alpha} \right\}_{\alpha \in \Lambda}$ denote the collection of ideals of $R$ that
contain $S$.  Prove the equality $(S)R = \cap_{\alpha \in \Lambda} I_{\alpha}$.
\end{exercise}

\begin{exercise}
%Let $R$ be a commutative ring with identity, l
Let $S \sse R$, and set $I = (S)R$.
Prove that $I$ is the unique smallest ideal containing the set $S$.
\end{exercise}

%\begin{definition}[(finite) generating set]
%Let $R$ be a commutative ring with identity and let $I \sse R$ be an ideal of $R$.
%Let $S \sse R$.  We say that $S$ is \emph{a generating set for $I$} if $I = \left( S
%\right)R$; in this case, we also say that \emph{$S$ generates $I$}, or that \emph{$I$
%is generated by $S$.} If $I$ has a finite generating set, then $I$ is said to be
%\emph{finitely generated}. In the special case where $I$ may be generated by one
%element, $I$ is said to be a \emph{principal ideal}.
%\end{definition}

\begin{exercise}
\begin{enumerate}[a.]
\item Give an example to show that
%if $I$ and $J$ are ideals of a ring $R$, then
$I \cup J$ need not be an ideal.
\item %Let $R$ be a commutative ring with identity; let $I$ and $J$ be ideals of $R$.
Prove that $I \cup J$ is an ideal of $R$ if and only if either $I \sse J$ or $J \sse
I$.
\end{enumerate}
\end{exercise}

\begin{exercise}
%Let $R$ be a commutative ring with identity.
Let $I$ and $J$ be ideals of $R$.
Set
\[
I + J = \left\{ a + b \mid a \in I, b \in J \right\}.
\]
Prove that $I + J$ is an ideal of $R$ and that $I\cup J\sse I+J$.
\end{exercise}

\begin{exercise}
%Let $R$ be a commutative ring with identity and let $I$ and $J$ be ideals of $R$.
Show that $I + J$ is the unique smallest ideal containing $I\cup J$.
\end{exercise}

\begin{exercise}
Let $I = (f_{1}, \ldots, f_{n})R$ and let $J = (g_{1}, \ldots, g_{m})R$.  Prove that
$I + J$ is generated by the set $\{f_{1}, \ldots, f_{n}, g_{1}, \ldots, g_{m}\}$.
\end{exercise}

\begin{exercise} % [Cox]
Determine whether the given polynomial in is in the given ideal $I \subseteq \mathbb R[x]$.
\begin{tabular}{l l }
\end{tabular}
\begin{enumerate} \item $f(x) = x^2 - 3x + 2, \hskip.25in I = <x-2>$ 

\item $f(x) = x^5 - 4x + 1, \hskip.25in I = <x^3-x^2+x>$ 

\item $f(x) = x^2 - 4x + 4, \hskip.25in I = <x^4 - 6x^2 + 12x - 8, 2x^3 - 10x^2 + 16x -8>$ 

\item $f(x) = x^3 - 1, \hskip.25in I = <x^9-1, x^5 + x^3 - x^2 -1>$ 
\end{enumerate}



\end{exercise}



\begin{exercise}\label{Exercise:IsAnIdeal?}
Let $R = \mathbb{Z}[x]$.  Prove or disprove:
\begin{enumerate}
\item The set $K$ of all constant polynomials in $R$ is an ideal
of $R$.
\item The set $I$ of all polynomials in $R$ with even constant terms is an ideal of
$R$.
\item The set $I$ of all polynomials in $R$ with odd constant terms is an ideal of
$R$.
\end{enumerate}
\end{exercise}


\begin{exercise}
Use Exercise~\ref{Exercise:IdealContainment} to prove the following equalities in the
polynomial ring $R = \mathbb{Q}[x, y]$:
\begin{enumerate}[a.]
\item $(x + y, x - y)R = (x, y)R$.

\item $(x + x y, y + x y, x^{2}, y^{2})R = (x, y)R$.

\item $(2x^{2} + 3y^{2} - 11, x^{2} - y^{2} - 3)R = (x^{2} - 4, y^{2} - 1)R$.
\end{enumerate}
This illustrates that the same ideal can have many different generating sets and that
different generating sets may have different numbers of elements.
\end{exercise}


\begin{exercise} % [RS]
Let $A$ be a commutative ring with identity.
Let $f$ and $g$ be monomials in $R=A[x_1,\ldots,x_d]$.  If
$(f)R=(g)R$, show that $f=g$.
\end{exercise}


\begin{exercise} % [RS]
Let $f$ be a monomial in $R=A[x_1,\dots,x_d]$ and let $n$ be an integer, $n \geq 1$.
Prove that $\deg(f) \leq n$ if and only if there exists a monomial $g$
of degree $n$ such that $g \in (f)R$.
\end{exercise}



\begin{exercise}  Let $x$ be a nilpotent element of a ring $A$.  Show that $1+x$ is a unit of $A$.  Deduce that the sum of a nilpotent element and a unit is a unit.
\end{exercise}

\begin{exercise} %[AM]
\label{polyring} Let $A$ be a ring and let $A[x]$ be the ring of polynomials in an indeterminate $x$, with coefficients in $A$.  Let $f = a_0 + a_1x + \dots + a_nx^n \in A[x]$.  Prove that:

\begin{enumerate} \item $f$ is a unit in $A[x] \Leftrightarrow a_0$ is a unit in $A$ and $a_1, a_2, \dots, a_n$ are nilpotent.  [HINT: If $b_0 + b_1x + \dots + b_mx^m$ is the inverse of $f$, prove by induction on $r$ that $a_n^{r+1}b_{m-r} = 0$.  Hence show that  $a_n$ is nilpotent and use the previous exercise.

\item $f$ is nilpotent $\Leftrightarrow a_0, a_1, \dots, a_n$ are nilpotent. [HINT: choose a monomial ordering and argue by induction on the number of summands.] 

\item $f$ is a zero-divisor $\Leftrightarrow$ there exists $a \neq 0$ in $A$ such that $af = 0$.  [HINT: Choose a polynomial $g = b_0 + b_1x + \dots + b_mx^m$ of least degree $m$ such that $fg = 0$.  Then $a_nb_m = 0$, hence $a_ng = 0$ (because $a_ng$ annihilates $f$ and has degree $<m$).  Now show by induction that $a_{n-r}g = 0$  $(0 \leq r \leq n)$.]

\end{enumerate}

\end{exercise} 



\end{document}