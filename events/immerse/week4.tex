\documentclass{amsart}

%\usepackage{doublespace}
\usepackage{amsmath, amsfonts, amssymb, amsthm, amscd}

%\usepackage[all]{xy}
\usepackage{enumerate}
\usepackage{graphicx}
%\usepackage{pdfsync}

\usepackage{vmargin}
\setpapersize{USletter}
\setmargrb{1in}{1in}{1in}{1in}


\setlength{\parindent}{0in}
\setlength{\parskip}{2mm}


\theoremstyle{plain}
\newtheorem{theorem}{Theorem}[section]
\newtheorem{lemma}[theorem]{Lemma}
\newtheorem{corollary}[theorem]{Corollary}
\newtheorem{proposition}[theorem]{Proposition}
\newtheorem{conjecture}[theorem]{Conjecture}
\newtheorem{principle}[theorem]{Principle}
\newtheorem{claim}[theorem]{Claim}

\theoremstyle{definition}
\newtheorem{definition}{Definition}
\newtheorem{example}[theorem]{Example}
\newtheorem{answer}[theorem]{Answer}
\newtheorem{remark}[theorem]{Remark}


\newtheorem{exercise}[theorem]{Exercise}
\newtheorem{paperexercise}[theorem]{Paper Exercise}



\theoremstyle{definition}
%\newtheorem{Definition}[Theorem][section]

%\newtheorem{example}{Example}[section]
%\newtheorem{definition}[theorem]{Definition}

%\input xy
%\xyoption{all}



\newcommand{\defining}[1]{\textbf{#1}}


%\newcommand{\sse}{\subseteq}
%\newcommand{\ssne}{\subsetneq}
\renewcommand{\colon}[3]{\ensuremath{\left(#1:_{#2} #3\right)}}
%\newcommand{\rad}[1]{\ensuremath{\operatorname{rad}\left(#1\right)}}
\DeclareMathOperator{\radname}{rad}
\newcommand{\rad}[1]{\radname(#1)}

\newcommand{\m}{\mathfrak{m}}

\newcommand{\N}{\mathbb{N}}
\newcommand{\Z}{\mathbb{Z}}
\newcommand{\Q}{\mathbb{Q}}
\newcommand{\R}{\mathbb{R}}


\DeclareMathOperator{\interior}{int}
\newcommand{\boundary}{\partial}

\newcommand{\tropicalsemiring}{\textbf{R}}
\newcommand{\troplus}{\oplus}
\newcommand{\tromult}{\odot}

\DeclareMathOperator{\aff}{aff}
\DeclareMathOperator{\conv}{conv}
\DeclareMathOperator{\relint}{relint}

%%%%%%%%%%%%%%%%%%%%%%%%%%%%%%%%%%%%%%%%%%%%%%%%%%%%%%%%%%%%%%%%%%%%%%%%%%%%
\begin{document}

\setcounter{page}{8}

\centerline{\large{\bf IMMERSE 2007}}
\vskip.25in

\centerline{\large{Algebra Exercises}}
\vskip.25in

\setcounter{section}{3}
\section{Week 4}

For the following exercise set, the ``Newton polytope (unbounded) of $I$'' is the UNBOUNDED convex hull
of all the monomials in $I$. It is what H\"ubl denotes $K(I)$.


\begin{exercise}
Show that if $I$ and $J$ are monomial ideals,
then so are: $I+J$, $I \cap J$, $IJ$, and $\rad{I}$.
\end{exercise}



\begin{exercise}
Let $I=(X^3,Y^2)$. Let $\m=(X,Y)$.
Find the set of monomials that are in $I$ and not in $\m I$.
(That is, the set of monomials in $I \setminus \m I$.)
Repeat for $I=(X^3,X^2Y,Y^2)$.
\end{exercise}

\begin{exercise}
Let $I \subset R = k[x_1,\dots,x_d]$ be a monomial ideal.
Let $\m = (x_1,\dots,x_d)$.
In Exercise~3.6 you showed that there is one and only one
irredundant generating set for $I$ consisting of monomials.
Show that this irredundant monomial generating set is given by
the set of monomials that are in $I$ but not in $\m I$.
\end{exercise}


\begin{exercise}
Let $I$ be an ideal (not necessarily monomial).
Show that for $a, b \geq 1$, we have $I^a I^b = I^{a+b}$.
The zeroth power is defined as $I^0 = (1)$, the unit ideal.
Show that $I^0 I^b = I^b = I^b I^0$.
\end{exercise}





%\begin{exercise}  Set $I = (X_0^4, X_0^3X_1^3, X_1^4) \subseteq k[X_0, X_1]$.
%Graph the Newton polytope (unbounded) of $I$ in $\mathbb R^2$.  
%\end{exercise}

\begin{exercise}  Set $I = (X_0^3X_1^5, X_0^4X_1^4, X_0^5X_1^2) \subseteq k[X_0, X_1]$.
Graph the Newton polytope (unbounded) of $I$ in $\mathbb R^2$.  
\end{exercise}

\begin{exercise}  Set $I =   (X_0^3X_1^5, X_0^4X_1^4, X_0^5X_1^2) \subseteq k[X_0, X_1, X_2]$.
Graph the Newton polytope (unbounded) of $I$ in $\mathbb R^3$.  
\end{exercise}





\begin{paperexercise}  For the the monomial ideal $I = (X_0^3X_1^5, X_0^4X_1^4, X_0^5X_1^2) \subseteq k[X_0, X_1, X_2]$, determine the following

(i) $C$, $K$, $\sigma_i$, $s_i$, $\Sigma_i$, and $\boundary K$, as in Lemma 1.

(ii) $S$, $A_j$, $\boundary \Sigma_i$, and $A_j \cap \boundary \Sigma_i$ as in Lemma 2.

(iii) $F_i$ as in the definition on page 3775.
\end{paperexercise}






\begin{exercise}
For this exercise, let $V$ be a vector space over the real numbers $\R$.

Recall the following definition: an \defining{affine subspace} $A \subset V$ is a translation of a vector
subspace, $A = L + b$ for some linear subspace (aka vector subspace) $L\subset V$ and some vector $b \in V$.

Show the following:
\begin{enumerate}[a.]
%\item For an affine subspace $A = L+b$, the subspace $L$ is unique.
%That is, if $A = L+b = L'+b'$, then $L=L'$.
%\item For an affine subspace $A = L+b$, we have $b \in A$.
%Conversely if $a \in A$ is any element, then $A = L+a$.
%Thus $A = L+x$ if and only if $x \in A$.
\item The Minkowski sum of two affine subspaces is again an affine subspace.
So is their convex hull.
\item An arbitrary intersection of affine subspaces is again an affine subspace; or empty.
\item Let $S \subset V$ be any nonempty subset. The \defining{affine hull of $S$} is defined as:
\[ \aff(S) = \{ v \in V : v = t_1 s_1 + \dots + t_d s_d, d \geq 1, s_i \in S, t_i \in \R, \sum t_i = 1 \} \]
%First, for $S = \{(1,0),(0,1)\} \subset \R^2$, find $\aff(S)$.
Show $\aff(S)$ is an affine subspace.
\item Let $S \subset V$ be any nonempty subset. Show that $\aff(S)$ as defined above
is equal to the intersection of all the affine subspaces of $V$ containing $S$.
\item Let $S \subset V$ be any nonempty subset and let $s \in S$.
Let $S - s$ be the translation of $S$ by $-s$. Let $L$ be the linear span of $S-s$.
Show that $\aff(S) = L+s$.
\end{enumerate}
\end{exercise}


%\begin{exercise}
%We defined ``the ideal generated by $S$'' to be the set of certain finite sums, and showed it is
%the intersection of all the ideals containing $S$.
%We defined ``the convex hull of $S$'' to be the set of certain finite sums, and showed it is
%the intersection of all the convex sets containing $S$.
%We defined ``the affine hull of $S$'' to be the set of certain finite sums, and showed it is
%the intersection of all the affine subspaces containing $S$.
%In each case, there is an ``internal description'' (in terms of certain sums that build the set up using $S$)
%and an ``external description'' (in terms of intersections of things that contain $S$)
%
%No exercise here, just wanted to point this out.
%\end{exercise}


\begin{exercise}
Recall we defined the dimension of an affine subspace $\dim A$ to be $\dim L$.
Show the following.
\begin{enumerate}[a.]
\item If there is a chain of affine subspaces
\[ A_0 \subsetneq A_1 \subsetneq \dots \subsetneq A_r = A, \]
then $\dim A \geq r$.
\item Furthermore, $\dim A = r$ where $r$ is the largest integer
such that there exists a chain of length $r$, as written above.
\item If $S \subset \R^n$ is a finite set with $p$ elements, then $\dim \aff(S) < p$.
\item We say $S$ is \defining{affinely independent} if $S$ is a finite set
with $p$ elements and $\dim \aff(S) = p-1$.
Now, if $A \subset \R^n$ is an affine subspace, show that $\dim A$
is one less than the size of the largest affinely independent subset of $A$.
\end{enumerate}
\end{exercise}


\begin{definition}
Let $R$ be a commutative ring with $1$.
A \defining{monic polynomial} with coefficients in $R$ is a polynomial
whose leading coefficient is $1$, that is, a polynomial
$ f = x^n + r_1 x^{n-1} + \dots + r_n $.
\end{definition}

\begin{exercise}
This exercise is an introduction to the theory of ``integral closure'' that we will talk about
in class soon. For this exercise, let $R = \Z$, the ring of integers with the usual operations.
\begin{enumerate}[a.]
\item Let $f(x) = x^n + r_1 x^{n-1} + \dots + r_n$ be a monic polynomial with coefficients $r_i \in \Z$.
Suppose $a/b \in \Q$ is a rational number such that $f(a/b) = 0$.
Then show $a/b \in \Z$.
[HINT: Say $a/b$ is written in lowest terms, ie, $a$ and $b$ have no common factors.
Say $p$ is a prime factor of $b$. Clear denominators in the equation $f(a/b)=0$,
then show $p$ is also a prime factor of $a$. Now, what is $b$?]
\item On the other hand, there are lots of irrational real numbers that satisfy monic polynomials with
integer coefficients. Show that $\sqrt{2}$ and $\sqrt{2+\sqrt[5]{3}}$ satisfy this condition.
Show that $i = \sqrt{-1}$ satisfies this condition.
Show that $\sqrt{2}+\sqrt{3}$ satisfies this condition.
[That is, find monic polynomials that these numbers satisfy.]
\item Show that $1/\sqrt{2}$ does NOT satisfy any monic polynomial with integer coefficients.
%(By the way, it turns out $\pi$ and $e$ do NOT satisfy monic polynomials with integer coefficients,
%so numbers like $\sqrt{2}$ are special in a way.
%If you are curious you can learn about this
%in books such as ``Making Transcendence Transparent'' by Edward Burger and Robert Tubbs.)
\end{enumerate}
\end{exercise}


\begin{exercise}
Show the following.
\begin{enumerate}[a.]
\item Show $(x^2 y^2)^r \in (x^4,x^3 y^3, y^4)^r$ for some $r > 0$.
\item Say $I$ is a monomial ideal generated by $m_1, \dots, m_t$,
with corresponding exponent vectors $v_1, \dots, v_t$.
Say $n$ is a monomial with exponent vector $w$.
Suppose $n^r \in I^r$ for some $r > 0$.
Then show $w \in K(I)$.
[HINT: Write $n^r = g \cdot m_1^{a_1} \dots m_t^{a_t}$
for some $g \in R$ and $\sum a_i = r$.
Show $g$ is a monomial; say its exponent vector is $u$.
Then $rw = u + a_1 v_1 + \dots + a_t v_t$. Divide by $r$.]
\item Let $I$, $m_i$, $v_i$, $n$, and $w$ be as before.
Suppose $w \in K(I)$.
Then show $n^r \in I^r$ for some $r > 0$.
[HINT: Use the fact that we can write $w = c_1 v_1 + \dots + c_t v_t$
with \textbf{rational} $c_i$, $\sum c_i \geq 1$.
Then clear denominators.]
\item For a monomial ideal $I$ and a monomial $n$,
we have $n^r \in I^r$ for some $r > 0$
if and only if $n \in K(I)$.
\end{enumerate}
\end{exercise}



\begin{definition}
A polynomial $f \in k[x_1,\dots,x_n]$ is called \defining{homogeneous of degree $d$}
if every term in $f$ has degree $d$.
For a polynomial $f$, we write $f = f_0 + f_1 + \dots + f_n$, where $f_d$ is the sum of all the
terms of degree $d$ in $f$; we say $f_d$ is the \defining{degree $d$ part of $f$}.
An ideal $I \subset k[x_1,\dots,x_n]$ is called a \defining{homogeneous ideal}
if there is a generating set for $I$ consisting of homogeneous polynomials.
\end{definition}

\begin{exercise}
Let $R = k[x_1,\dots,x_n]$.
Show the following.
\begin{enumerate}[a.]
%\item $f$ is homogeneous of degree $d$ if and only if
%$f( t x_1 , \dots , t x_n ) = t^d f(x_1, \dots, x_n)$ for all $t$.
%%%% Is that even true? What if k = F_p and d=p? Then t^d=t^p=t for all t... hmmm Or what if d=p-1?
\item If $f$ is homogeneous of degree $d$ then
 $d \cdot f = \sum_{i=1}^n x_i \frac{\partial f}{\partial x_i}$.
 Also $f(tx_1,\dots,tx_n) = t^d f(x_1,\dots,x_n)$.
%(But the converse fails: e.g., $x^p-x \in (\Z/p\Z)[x]$ satisfies this condition, but is not homogeneous.)
\item If $f$ is homogeneous of degree $d$ and $g$ is homogeneous of degree $e$,
then $fg$ is homogeneous of degree $d+e$.
\item If $I$ and $J$ are homogeneous ideals, then so are $I+J$, $I \cap J$, and $IJ$.
And so is $\rad{I}$.
\item Show that an ideal $I$ is homogeneous if and only if for every $f \in I$ and for every $d$,
the degree $d$ part $f_d \in I$. [Compare with Exercise~3.5.]
Show $f$ is homogeneous if and only if $(f)$ is homogeneous.
\item Let $I$ be a homogeneous ideal. Show that either $I = (1)$ or $I \subseteq (x_1,\dots,x_n)$.
\end{enumerate}
\end{exercise}



%\begin{exercise}
%Let $S \subset \R^n$ be a non-empty convex set.
%The \defining{recession cone of $S$} is the set $RC(S)$ defined by
%$ RC(S) = \{ v : S+v \subseteq S \} $.
%Show that $0 \in RC(S)$, that $RC(S)$ is closed under addition, and that $RC(S)$ is closed under scalar multiplication.
%Show $RC(S)$ is convex.

%For $I$ a monomial ideal, show that the recession cone of the Newton polytope of $I$ is the positive orthant.
%\end{exercise}

\end{document}
