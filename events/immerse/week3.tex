\documentclass{amsart}

%\usepackage{doublespace}
\usepackage{amsmath, amsfonts, amssymb, amsthm, amscd}

%\usepackage[all]{xy}
\usepackage{enumerate}
\usepackage{graphicx}
%\usepackage{pdfsync}

\usepackage{vmargin}
\setpapersize{USletter}
\setmargrb{1in}{1in}{1in}{1in}


\setlength{\parindent}{0in}
\setlength{\parskip}{2mm}


\theoremstyle{plain}
\newtheorem{theorem}{Theorem}[section]
\newtheorem{lemma}[theorem]{Lemma}
\newtheorem{corollary}[theorem]{Corollary}
\newtheorem{proposition}[theorem]{Proposition}
\newtheorem{conjecture}[theorem]{Conjecture}
\newtheorem{principle}[theorem]{Principle}
\newtheorem{claim}[theorem]{Claim}

\theoremstyle{definition}
\newtheorem{definition}{Definition}
\newtheorem{example}[theorem]{Example}
\newtheorem{answer}[theorem]{Answer}
\newtheorem{remark}[theorem]{Remark}


\newtheorem{exercise}[theorem]{Exercise}
\newtheorem{paperexercise}[theorem]{Paper Exercise}



\theoremstyle{definition}
%\newtheorem{Definition}[Theorem][section]

%\newtheorem{example}{Example}[section]
%\newtheorem{definition}[theorem]{Definition}

%\input xy
%\xyoption{all}



\newcommand{\defining}[1]{\textbf{#1}}


%\newcommand{\sse}{\subseteq}
%\newcommand{\ssne}{\subsetneq}
\renewcommand{\colon}[3]{\ensuremath{\left(#1:_{#2} #3\right)}}
%\newcommand{\rad}[1]{\ensuremath{\operatorname{rad}\left(#1\right)}}
\DeclareMathOperator{\radname}{rad}
\newcommand{\rad}[1]{\radname(#1)}

\newcommand{\m}{\mathfrak{m}}

\newcommand{\N}{\mathbb{N}}
\newcommand{\Z}{\mathbb{Z}}
\newcommand{\Q}{\mathbb{Q}}
\newcommand{\R}{\mathbb{R}}


\DeclareMathOperator{\interior}{int}
\newcommand{\boundary}{\partial}

\newcommand{\tropicalsemiring}{\textbf{R}}
\newcommand{\troplus}{\oplus}
\newcommand{\tromult}{\odot}

%%%%%%%%%%%%%%%%%%%%%%%%%%%%%%%%%%%%%%%%%%%%%%%%%%%%%%%%%%%%%%%%%%%%%%%%%%%%
\begin{document}

\setcounter{page}{6}

\centerline{\large{\bf IMMERSE 2007}}
\vskip.25in

\centerline{\large{Algebra Exercises}}
\vskip.25in

\setcounter{section}{2}
\section{Week 3}








\begin{exercise} % [RS]
Let $f$ and $g$ be monomials in $R=k[x_1,\dots,x_n]$.
\begin{enumerate}[a.]
\item Say $f = x_1^{a_1} \cdots x_n^{a_n}$ and $g = x_1^{b_1} \cdots x_n^{b_n}$.
Show $f \mid g$ if and only if
for all $1 \leq i \leq n$, we have $a_i \leq b_i$.

\item Prove that if $f \in (g)R$, then $\deg(f) \geq \deg(g)$.

\item Is the converse true or false? Explain. [HINT: Treat the cases $n=1$ and $n>1$ separately.]

\item Prove that if $f$ and $g$ have the same degree
and if $f \in (g)R$, then $g = f$.
\end{enumerate}
\end{exercise}



\begin{definition}
Let $R$ be a commutative ring with unity, $S \subset R$, and $I = (S)$.
An element $s \in S$ is called a \defining{redundant} generator
if the set $S \setminus \{s\}$ generates the same ideal $I$.
If $S$ contains no redundant generators, then $S$ is called \defining{irredundant}.
\end{definition}

\begin{exercise}
In the ring $\Z$, show that $(6,10) = (2)$, but $\{6,10\}$
is an irredundant generating set for the ideal.
\end{exercise}

\begin{exercise} % [RS]
Let $z_1,\ldots,z_m$ be monomials in $R=k[x_1,\ldots,x_d]$ and set
$J=(z_1,\ldots,z_m)$.
Show that the following conditions are equivalent.
\begin{enumerate}[(i)]
\item
$z_1,\ldots,z_m$ is an irredundant generating sequence for $J$.
\item
For all $i\neq j$, we have $z_i\nmid z_j$.
\end{enumerate}
\end{exercise}

\begin{exercise} % [RS]
Set $R=k[x_1,\ldots,x_n]$ and $\m = (x_1,\dots,x_n)$,
and let $f\in R$.
Show that $f\in \m^n$ if and only if each monomial occurring in $f$
has degree $\geq n$.
\end{exercise}

\begin{exercise} % [RS]
Let $R=k[x_1,\ldots,x_d]$.
Show that the following conditions are equivalent.
\begin{enumerate}[(i)]
\item $I$ is a monomial ideal.
%\item $I$ is generated by monomials.
\item For each $f\in I$ each monomial occurring in $f$ with nonzero coefficient is in $I$.
\end{enumerate}
\end{exercise}


\begin{exercise}
Let $I \subset R = k[x_1,\dots,x_n]$ be a monomial ideal. Show the following:
There is a unique irredundant monomial generating set for $I$.
It is the unique containment-minimal monomial generating set.
A set of monomials in $I$ generates $I$ if and only if it contains this minimal set.
\end{exercise}

\begin{exercise} % [M.~Artin]
Let $I, J$, and $J'$ be ideals in a ring $R$.  Prove or disprove: $I(J + J') = IJ + IJ'$.
\end{exercise}

\begin{exercise} % [M.~Artin]
Let $I$ and $J$ be ideals of $R$ such that $I+J = R$.  Prove that $IJ = I \cap J$.
\end{exercise}

%\begin{exercise} % [Cox]
%Let $I \subseteq k[x,y]$ be the monomial ideal spanned over $k$
%by the monomials $x^{\beta}$ corresponding to $\beta$ in the shaded region below:
%\vskip.25in

%\begin{enumerate}  \item (Use the method given in the proof of Theorem 5 to) find an ideal basis for $I$.

%\item Is your basis as small as possible, or can some $\beta$'s be deleted from your basis, yielding a smaller set that generates the same ideal?
%\end{enumerate}
%\end{exercise}

\begin{exercise}
Prove or disprove:
\begin{enumerate}[a.]
\item Every  (finite or infinite) intersection of convex sets is convex.
\item A finite union of convex sets is convex.
\item The convex hull of a finite collection of convex sets is the smallest convex set
containing the union of the collection.
\item If $K_i$ is a convex set for $i=1,2,3,\dots$ and $K_1 \subset K_2 \subset \dots$
then the union $\bigcup K_i$ is convex.
\end{enumerate}
(Compare with exercises 1.5, 1.10, 1.12, and 2.19.)
The above statements suggest an analogy between ideals and convex sets.
For each of the following operations on ideals, identify the analogous operation on convex sets.
How many more statements about ideals or convex sets can you come up with, based on this analogy?
\begin{enumerate}[i.]
\item The intersection of ideals.
\item The sum of ideals.
\item The ideal generated by a set.
\end{enumerate}
\end{exercise}


\begin{exercise}
Prove:
If $K \subseteq R^d$ is convex, $x \in K$ and $y \in \interior(K)$, then all points of the line segment between $x$ and $y$ belong to $\interior(K)$.
\end{exercise}


\begin{definition}
Let $I \subset R = k[x_1,\dots,x_n]$ be a monomial ideal.
The \defining{Newton polytope of $I$} is the convex hull in $\R^n$ of the set
of exponent vectors of all the monomials in $I$.
The \defining{Newton polyhedron of $I$} is the convex hull in $\R^n$ of the set
of exponent vectors of the monomials in a minimal generating set for $I$.

In the paper we are reading, H\"ubl refers to the Newton polytope of $I$ as $K(I)$, or $K$,
and to the Newton polyhedron of $I$ as $C(I)$, or $C$.
\end{definition}


\begin{exercise}
Let $I$, $J$ be monomial ideals.
Express the following in terms of the Newton polytopes of $I$ and of $J$:
\begin{enumerate}[a.]
\item The Newton polytope of $I+J$
\item ... of $I J$
\item ... of $I \cap J$
\end{enumerate}
\end{exercise}

\begin{exercise}
For the following ideals, find the facets (codimension-$1$ faces) of the Newton polytopes.
Which facets are unbounded?
\begin{enumerate}[a.]
\item $I = (x,y) \subset k[x,y,z]$
\item $I = (x^2,y^2,z^2) \subset k[x,y,z]$
\item $I = (x^2,y^2,z^2,xyz) \subset k[x,y,z]$
\end{enumerate}
\end{exercise}

\begin{paperexercise}
For the the monomial ideal $I = (X_0^4, X_0^3X_1^3, X_1^4)$
(as on page 3772 of the H\"ubl paper), determine $S$, $A_j$, $\delta \sum_i$,
and $A_j \cap \delta \sum_i$ as in Lemma 2 and $F_i$ as in the definition on page 3775.
\end{paperexercise}

\begin{paperexercise}
For the the monomial ideal
$I = (X_0^3X_1^5, X_0^4X_1^4, X_0^5X_1^2) \subseteq k[X_0, X_1]$,
determine the following:
\begin{enumerate}[a.]
\item $C$, $K$, $\sigma_i$, $s_i$, $\sum_i$, and $\boundary K$, as in Lemma 1.
\item $S$, $A_j$, $\delta \sum_i$, and $A_j \cap \boundary \sum_i$ as in Lemma 2
\item $F_i$ as in the definition on page 3775.
\end{enumerate}
\end{paperexercise}


\begin{definition}
Let $S$ be a set and $\prec$ a binary relation on $S$
such that $a \not \prec a$ for all $a \in S$.
We define a new relation, ``weak $\prec$'', denoted $\preceq$, by
$a \preceq b$ if and only if either $a \prec b$ or $a=b$.
Then $\prec$ is a \defining{(strict) partial order} and $\preceq$ is a \defining{(weak) partial order}
if and only if $\preceq$ is antisymmetric and transitive.
(Recall $\preceq$ is antisymmetric if $a \preceq b$ and $b \preceq a$ implies $a=b$,
and $\preceq$ is antisymmetric if $a \preceq b$ and $b \preceq c$ implies $a \preceq c$.)

Furthermore $\prec$ and $\preceq$ are \defining{total orders}
if in addition to the above, for every $a, b \in S$, either $a \preceq b$ or $b \preceq a$.
Equivalently, either $a \prec b$, $a=b$, or $b \prec a$.
\end{definition}

\begin{exercise} % [RS]
\begin{enumerate}[a.]
\item
Show that the lexicographical ordering $<$ on the monomials in $R=A[x,y]$ is a total ordering.
Also, for each monomial $f\in R$, show that $f\not< f$.
\item
Define a lexicographical ordering on the monomials in $A[x_1,\ldots,x_d]$ and prove that
it satisfies the properties from part a.
\end{enumerate}
\end{exercise}

\begin{exercise} % [Rotman 6.76]
Write the first 10 monic monomials in $k[x,y]$ in lexicographic order and in degree-lexicographic order.
\end{exercise}

\begin{exercise} % [Rotman 6.76]
Write all the monic monomials in $k[x,y,z]$ of total degree at most $2$
in lexicographic order and in degree-lexicographic order.
\end{exercise}





\end{document}
