%********************************************%
%*       Generated from PreTeXt source      *%
%*       on 2020-08-06T14:40:39-06:00       *%
%*   A recent stable commit (2020-07-07):   *%
%* 1771dacf84bfb1789139598d8379414b560b8f17 *%
%*                                          *%
%*         https://pretextbook.org          *%
%*                                          *%
%********************************************%
\documentclass[oneside,10pt,]{article}
%% Custom Preamble Entries, early (use latex.preamble.early)
%% Default LaTeX packages
%%   1.  always employed (or nearly so) for some purpose, or
%%   2.  a stylewriter may assume their presence
\usepackage{geometry}
%% Some aspects of the preamble are conditional,
%% the LaTeX engine is one such determinant
\usepackage{ifthen}
%% etoolbox has a variety of modern conveniences
\usepackage{etoolbox}
\usepackage{ifxetex,ifluatex}
%% Raster graphics inclusion
\usepackage{graphicx}
%% Color support, xcolor package
%% Always loaded, for: add/delete text, author tools
%% Here, since tcolorbox loads tikz, and tikz loads xcolor
\PassOptionsToPackage{usenames,dvipsnames,svgnames,table}{xcolor}
\usepackage{xcolor}
%% begin: defined colors, via xcolor package, for styling
%% end: defined colors, via xcolor package, for styling
%% Colored boxes, and much more, though mostly styling
%% skins library provides "enhanced" skin, employing tikzpicture
%% boxes may be configured as "breakable" or "unbreakable"
%% "raster" controls grids of boxes, aka side-by-side
\usepackage{tcolorbox}
\tcbuselibrary{skins}
\tcbuselibrary{breakable}
\tcbuselibrary{raster}
%% We load some "stock" tcolorbox styles that we use a lot
%% Placement here is provisional, there will be some color work also
%% First, black on white, no border, transparent, but no assumption about titles
\tcbset{ bwminimalstyle/.style={size=minimal, boxrule=-0.3pt, frame empty,
colback=white, colbacktitle=white, coltitle=black, opacityfill=0.0} }
%% Second, bold title, run-in to text/paragraph/heading
%% Space afterwards will be controlled by environment,
%% independent of constructions of the tcb title
%% Places \blocktitlefont onto many block titles
\tcbset{ runintitlestyle/.style={fonttitle=\blocktitlefont\upshape\bfseries, attach title to upper} }
%% Spacing prior to each exercise, anywhere
\tcbset{ exercisespacingstyle/.style={before skip={1.5ex plus 0.5ex}} }
%% Spacing prior to each block
\tcbset{ blockspacingstyle/.style={before skip={2.0ex plus 0.5ex}} }
%% xparse allows the construction of more robust commands,
%% this is a necessity for isolating styling and behavior
%% The tcolorbox library of the same name loads the base library
\tcbuselibrary{xparse}
%% Hyperref should be here, but likes to be loaded late
%%
%% Inline math delimiters, \(, \), need to be robust
%% 2016-01-31:  latexrelease.sty  supersedes  fixltx2e.sty
%% If  latexrelease.sty  exists, bugfix is in kernel
%% If not, bugfix is in  fixltx2e.sty
%% See:  https://tug.org/TUGboat/tb36-3/tb114ltnews22.pdf
%% and read "Fewer fragile commands" in distribution's  latexchanges.pdf
\IfFileExists{latexrelease.sty}{}{\usepackage{fixltx2e}}
%% Text height identically 9 inches, text width varies on point size
%% See Bringhurst 2.1.1 on measure for recommendations
%% 75 characters per line (count spaces, punctuation) is target
%% which is the upper limit of Bringhurst's recommendations
\geometry{letterpaper,total={340pt,9.0in}}
%% Custom Page Layout Adjustments (use latex.geometry)
%% This LaTeX file may be compiled with pdflatex, xelatex, or lualatex executables
%% LuaTeX is not explicitly supported, but we do accept additions from knowledgeable users
%% The conditional below provides  pdflatex  specific configuration last
%% begin: engine-specific capabilities
\ifthenelse{\boolean{xetex} \or \boolean{luatex}}{%
%% begin: xelatex and lualatex-specific default configuration
\ifxetex\usepackage{xltxtra}\fi
%% realscripts is the only part of xltxtra relevant to lualatex 
\ifluatex\usepackage{realscripts}\fi
%% end:   xelatex and lualatex-specific default configuration
}{
%% begin: pdflatex-specific default configuration
%% We assume a PreTeXt XML source file may have Unicode characters
%% and so we ask LaTeX to parse a UTF-8 encoded file
%% This may work well for accented characters in Western language,
%% but not with Greek, Asian languages, etc.
%% When this is not good enough, switch to the  xelatex  engine
%% where Unicode is better supported (encouraged, even)
\usepackage[utf8]{inputenc}
%% end: pdflatex-specific default configuration
}
%% end:   engine-specific capabilities
%%
%% Fonts.  Conditional on LaTex engine employed.
%% Default Text Font: The Latin Modern fonts are
%% "enhanced versions of the [original TeX] Computer Modern fonts."
%% We use them as the default text font for PreTeXt output.
%% Default Monospace font: Inconsolata (aka zi4)
%% Sponsored by TUG: http://levien.com/type/myfonts/inconsolata.html
%% Loaded for documents with intentional objects requiring monospace
%% See package documentation for excellent instructions
%% fontspec will work universally if we use filename to locate OTF files
%% Loads the "upquote" package as needed, so we don't have to
%% Upright quotes might come from the  textcomp  package, which we also use
%% We employ the shapely \ell to match Google Font version
%% pdflatex: "varl" package option produces shapely \ell
%% pdflatex: "var0" package option produces plain zero (not used)
%% pdflatex: "varqu" package option produces best upright quotes
%% xelatex,lualatex: add OTF StylisticSet 1 for shapely \ell
%% xelatex,lualatex: add OTF StylisticSet 2 for plain zero (not used)
%% xelatex,lualatex: add OTF StylisticSet 3 for upright quotes
%%
%% Automatic Font Control
%% Portions of a document, are, or may, be affected by defined commands
%% These are perhaps more flexible when using  xelatex  rather than  pdflatex
%% The following definitions are meant to be re-defined in a style, using \renewcommand
%% They are scoped when employed (in a TeX group), and so should not be defined with an argument
\newcommand{\divisionfont}{\relax}
\newcommand{\blocktitlefont}{\relax}
\newcommand{\contentsfont}{\relax}
\newcommand{\pagefont}{\relax}
\newcommand{\tabularfont}{\relax}
\newcommand{\xreffont}{\relax}
\newcommand{\titlepagefont}{\relax}
%%
\ifthenelse{\boolean{xetex} \or \boolean{luatex}}{%
%% begin: font setup and configuration for use with xelatex
%% Generally, xelatex is necessary for non-Western fonts
%% fontspec package provides extensive control of system fonts,
%% meaning *.otf (OpenType), and apparently *.ttf (TrueType)
%% that live *outside* your TeX/MF tree, and are controlled by your *system*
%% (it is possible that a TeX distribution will place fonts in a system location)
%%
%% The fontspec package is the best vehicle for using different fonts in  xelatex
%% So we load it always, no matter what a publisher or style might want
%%
\usepackage{fontspec}
%%
%% begin: xelatex main font ("font-xelatex-main" template)
%% Latin Modern Roman is the default font for xelatex and so is loaded with a TU encoding
%% *in the format* so we can't touch it, only perhaps adjust it later
%% in one of two ways (then known by NFSS names such as "lmr")
%% (1) via NFSS with font family names such as "lmr" and "lmss"
%% (2) via fontspec with commands like \setmainfont{Latin Modern Roman}
%% The latter requires the font to be known at the system-level by its font name,
%% but will give access to OTF font features through optional arguments
%% https://tex.stackexchange.com/questions/470008/
%% where-and-how-does-fontspec-sty-specify-the-default-font-latin-modern-roman
%% http://tex.stackexchange.com/questions/115321
%% /how-to-optimize-latin-modern-font-with-xelatex
%%
%% end:   xelatex main font ("font-xelatex-main" template)
%% begin: xelatex mono font ("font-xelatex-mono" template)
%% (conditional on non-trivial uses being present in source)
\IfFontExistsTF{Inconsolatazi4-Regular.otf}{}{\GenericError{}{The font "Inconsolatazi4-Regular.otf" requested by PreTeXt output is not available.  Either a file cannot be located in default locations via a filename, or a font is not known by its name as part of your system.}{Consult the PreTeXt Guide for help with LaTeX fonts.}{}}
\IfFontExistsTF{Inconsolatazi4-Bold.otf}{}{\GenericError{}{The font "Inconsolatazi4-Bold.otf" requested by PreTeXt output is not available.  Either a file cannot be located in default locations via a filename, or a font is not known by its name as part of your system.}{Consult the PreTeXt Guide for help with LaTeX fonts.}{}}
\usepackage{zi4}
\setmonofont[BoldFont=Inconsolatazi4-Bold.otf,StylisticSet={1,3}]{Inconsolatazi4-Regular.otf}
%% end:   xelatex mono font ("font-xelatex-mono" template)
%% begin: xelatex font adjustments ("font-xelatex-style" template)
%% end:   xelatex font adjustments ("font-xelatex-style" template)
%%
%% Extensive support for other languages
\usepackage{polyglossia}
%% Set main/default language based on pretext/@xml:lang value
%% document language code is "en-US", US English
%% usmax variant has extra hypenation
\setmainlanguage[variant=usmax]{english}
%% Enable secondary languages based on discovery of @xml:lang values
%% Enable fonts/scripts based on discovery of @xml:lang values
%% Western languages should be ably covered by Latin Modern Roman
%% end:   font setup and configuration for use with xelatex
}{%
%% begin: font setup and configuration for use with pdflatex
%% begin: pdflatex main font ("font-pdflatex-main" template)
\usepackage{lmodern}
\usepackage[T1]{fontenc}
%% end:   pdflatex main font ("font-pdflatex-main" template)
%% begin: pdflatex mono font ("font-pdflatex-mono" template)
%% (conditional on non-trivial uses being present in source)
\usepackage[varqu,varl]{inconsolata}
%% end:   pdflatex mono font ("font-pdflatex-mono" template)
%% begin: pdflatex font adjustments ("font-pdflatex-style" template)
%% end:   pdflatex font adjustments ("font-pdflatex-style" template)
%% end:   font setup and configuration for use with pdflatex
}
%% Symbols, align environment, commutative diagrams, bracket-matrix
\usepackage{amsmath}
\usepackage{amscd}
\usepackage{amssymb}
%% allow page breaks within display mathematics anywhere
%% level 4 is maximally permissive
%% this is exactly the opposite of AMSmath package philosophy
%% there are per-display, and per-equation options to control this
%% split, aligned, gathered, and alignedat are not affected
\allowdisplaybreaks[4]
%% allow more columns to a matrix
%% can make this even bigger by overriding with  latex.preamble.late  processing option
\setcounter{MaxMatrixCols}{30}
%%
%%
%% Division Titles, and Page Headers/Footers
%% titlesec package, loading "titleps" package cooperatively
%% See code comments about the necessity and purpose of "explicit" option.
%% The "newparttoc" option causes a consistent entry for parts in the ToC 
%% file, but it is only effective if there is a \titleformat for \part.
%% "pagestyles" loads the  titleps  package cooperatively.
\usepackage[explicit, newparttoc, pagestyles]{titlesec}
%% The companion titletoc package for the ToC.
\usepackage{titletoc}
%% begin: customizations of page styles via the modal "titleps-style" template
%% Designed to use commands from the LaTeX "titleps" package
\pagestyle{plain}
%% end: customizations of page styles via the modal "titleps-style" template
%%
%% Create globally-available macros to be provided for style writers
%% These are redefined for each occurence of each division
\newcommand{\divisionnameptx}{\relax}%
\newcommand{\titleptx}{\relax}%
\newcommand{\subtitleptx}{\relax}%
\newcommand{\shortitleptx}{\relax}%
\newcommand{\authorsptx}{\relax}%
\newcommand{\epigraphptx}{\relax}%
%% Create environments for possible occurences of each division
%% Environment for a PTX "section" at the level of a LaTeX "section"
\NewDocumentEnvironment{sectionptx}{mmmmmm}
{%
\renewcommand{\divisionnameptx}{Section}%
\renewcommand{\titleptx}{#1}%
\renewcommand{\subtitleptx}{#2}%
\renewcommand{\shortitleptx}{#3}%
\renewcommand{\authorsptx}{#4}%
\renewcommand{\epigraphptx}{#5}%
\section[{#3}]{#1}%
\label{#6}%
}{}%
%% Environment for a PTX "subsection" at the level of a LaTeX "subsection"
\NewDocumentEnvironment{subsectionptx}{mmmmmm}
{%
\renewcommand{\divisionnameptx}{Subsection}%
\renewcommand{\titleptx}{#1}%
\renewcommand{\subtitleptx}{#2}%
\renewcommand{\shortitleptx}{#3}%
\renewcommand{\authorsptx}{#4}%
\renewcommand{\epigraphptx}{#5}%
\subsection[{#3}]{#1}%
\label{#6}%
}{}%
%% Environment for a PTX "subsubsection" at the level of a LaTeX "subsubsection"
\NewDocumentEnvironment{subsubsectionptx}{mmmmmm}
{%
\renewcommand{\divisionnameptx}{Subsubsection}%
\renewcommand{\titleptx}{#1}%
\renewcommand{\subtitleptx}{#2}%
\renewcommand{\shortitleptx}{#3}%
\renewcommand{\authorsptx}{#4}%
\renewcommand{\epigraphptx}{#5}%
\subsubsection[{#3}]{#1}%
\label{#6}%
}{}%
%%
%% Styles for six traditional LaTeX divisions
\titleformat{\part}[display]
{\divisionfont\Huge\bfseries\centering}{\divisionnameptx\space\thepart}{30pt}{\Huge#1}
[{\Large\centering\authorsptx}]
\titleformat{\chapter}[display]
{\divisionfont\huge\bfseries}{\divisionnameptx\space\thechapter}{20pt}{\Huge#1}
[{\Large\authorsptx}]
\titleformat{name=\chapter,numberless}[display]
{\divisionfont\huge\bfseries}{}{0pt}{#1}
[{\Large\authorsptx}]
\titlespacing*{\chapter}{0pt}{50pt}{40pt}
\titleformat{\section}[hang]
{\divisionfont\Large\bfseries}{\thesection}{1ex}{#1}
[{\large\authorsptx}]
\titleformat{name=\section,numberless}[block]
{\divisionfont\Large\bfseries}{}{0pt}{#1}
[{\large\authorsptx}]
\titlespacing*{\section}{0pt}{3.5ex plus 1ex minus .2ex}{2.3ex plus .2ex}
\titleformat{\subsection}[hang]
{\divisionfont\large\bfseries}{\thesubsection}{1ex}{#1}
[{\normalsize\authorsptx}]
\titleformat{name=\subsection,numberless}[block]
{\divisionfont\large\bfseries}{}{0pt}{#1}
[{\normalsize\authorsptx}]
\titlespacing*{\subsection}{0pt}{3.25ex plus 1ex minus .2ex}{1.5ex plus .2ex}
\titleformat{\subsubsection}[hang]
{\divisionfont\normalsize\bfseries}{\thesubsubsection}{1em}{#1}
[{\small\authorsptx}]
\titleformat{name=\subsubsection,numberless}[block]
{\divisionfont\normalsize\bfseries}{}{0pt}{#1}
[{\normalsize\authorsptx}]
\titlespacing*{\subsubsection}{0pt}{3.25ex plus 1ex minus .2ex}{1.5ex plus .2ex}
\titleformat{\paragraph}[hang]
{\divisionfont\normalsize\bfseries}{\theparagraph}{1em}{#1}
[{\small\authorsptx}]
\titleformat{name=\paragraph,numberless}[block]
{\divisionfont\normalsize\bfseries}{}{0pt}{#1}
[{\normalsize\authorsptx}]
\titlespacing*{\paragraph}{0pt}{3.25ex plus 1ex minus .2ex}{1.5em}
%%
%% Styles for five traditional LaTeX divisions
\titlecontents{part}%
[0pt]{\contentsmargin{0em}\addvspace{1pc}\contentsfont\bfseries}%
{\Large\thecontentslabel\enspace}{\Large}%
{}%
[\addvspace{.5pc}]%
\titlecontents{chapter}%
[0pt]{\contentsmargin{0em}\addvspace{1pc}\contentsfont\bfseries}%
{\large\thecontentslabel\enspace}{\large}%
{\hfill\bfseries\thecontentspage}%
[\addvspace{.5pc}]%
\dottedcontents{section}[3.8em]{\contentsfont}{2.3em}{1pc}%
\dottedcontents{subsection}[6.1em]{\contentsfont}{3.2em}{1pc}%
\dottedcontents{subsubsection}[9.3em]{\contentsfont}{4.3em}{1pc}%
%%
%% Begin: Semantic Macros
%% To preserve meaning in a LaTeX file
%%
%% \mono macro for content of "c", "cd", "tag", etc elements
%% Also used automatically in other constructions
%% Simply an alias for \texttt
%% Always defined, even if there is no need, or if a specific tt font is not loaded
\newcommand{\mono}[1]{\texttt{#1}}
%%
%% Following semantic macros are only defined here if their
%% use is required only in this specific document
%%
%% Used for inline definitions of terms
\newcommand{\terminology}[1]{\textbf{#1}}
%% Titles of longer works (e.g. books, versus articles)
\newcommand{\pubtitle}[1]{\textsl{#1}}
%% End: Semantic Macros
%% Division Numbering: Chapters, Sections, Subsections, etc
%% Division numbers may be turned off at some level ("depth")
%% A section *always* has depth 1, contrary to us counting from the document root
%% The latex default is 3.  If a larger number is present here, then
%% removing this command may make some cross-references ambiguous
%% The precursor variable $numbering-maxlevel is checked for consistency in the common XSL file
\setcounter{secnumdepth}{3}
%%
%% AMS "proof" environment is no longer used, but we leave previously
%% implemented \qedhere in place, should the LaTeX be recycled
\newcommand{\qedhere}{\relax}
%%
%% A faux tcolorbox whose only purpose is to provide common numbering
%% facilities for most blocks (possibly not projects, 2D displays)
%% Controlled by  numbering.theorems.level  processing parameter
\newtcolorbox[auto counter, number within=section]{block}{}
%%
%% This document is set to number PROJECT-LIKE on a separate numbering scheme
%% So, a faux tcolorbox whose only purpose is to provide this numbering
%% Controlled by  numbering.projects.level  processing parameter
\newtcolorbox[auto counter, number within=section]{project-distinct}{}
%% A faux tcolorbox whose only purpose is to provide common numbering
%% facilities for 2D displays which are subnumbered as part of a "sidebyside"
\newtcolorbox[auto counter, number within=tcb@cnt@block, number freestyle={\noexpand\thetcb@cnt@block(\noexpand\alph{\tcbcounter})}]{subdisplay}{}
%%
%% tcolorbox, with styles, for FIGURE-LIKE
%%
%% tableptx: 2-D display structure
\tcbset{ tableptxstyle/.style={bwminimalstyle, middle=1ex, blockspacingstyle, coltitle=black, bottomtitle=2ex, titlerule=-0.3pt, fonttitle=\blocktitlefont} }
\newtcolorbox[use counter from=block]{tableptx}[3]{title={{\textbf{Table~\thetcbcounter}\space#1}}, phantomlabel={#2}, unbreakable, parbox=false, tableptxstyle, }
%%
%% xparse environments for introductions and conclusions of divisions
%%
%% introduction: in a structured division
\NewDocumentEnvironment{introduction}{m}
{\notblank{#1}{\noindent\textbf{#1}\space}{}}{\par\medskip}
%% Localize LaTeX supplied names (possibly none)
%% For improved tables
\usepackage{array}
%% Some extra height on each row is desirable, especially with horizontal rules
%% Increment determined experimentally
\setlength{\extrarowheight}{0.2ex}
%% Define variable thickness horizontal rules, full and partial
%% Thicknesses are 0.03, 0.05, 0.08 in the  booktabs  package
\newcommand{\hrulethin}  {\noalign{\hrule height 0.04em}}
\newcommand{\hrulemedium}{\noalign{\hrule height 0.07em}}
\newcommand{\hrulethick} {\noalign{\hrule height 0.11em}}
%% We preserve a copy of the \setlength package before other
%% packages (extpfeil) get a chance to load packages that redefine it
\let\oldsetlength\setlength
\newlength{\Oldarrayrulewidth}
\newcommand{\crulethin}[1]%
{\noalign{\global\oldsetlength{\Oldarrayrulewidth}{\arrayrulewidth}}%
\noalign{\global\oldsetlength{\arrayrulewidth}{0.04em}}\cline{#1}%
\noalign{\global\oldsetlength{\arrayrulewidth}{\Oldarrayrulewidth}}}%
\newcommand{\crulemedium}[1]%
{\noalign{\global\oldsetlength{\Oldarrayrulewidth}{\arrayrulewidth}}%
\noalign{\global\oldsetlength{\arrayrulewidth}{0.07em}}\cline{#1}%
\noalign{\global\oldsetlength{\arrayrulewidth}{\Oldarrayrulewidth}}}
\newcommand{\crulethick}[1]%
{\noalign{\global\oldsetlength{\Oldarrayrulewidth}{\arrayrulewidth}}%
\noalign{\global\oldsetlength{\arrayrulewidth}{0.11em}}\cline{#1}%
\noalign{\global\oldsetlength{\arrayrulewidth}{\Oldarrayrulewidth}}}
%% Single letter column specifiers defined via array package
\newcolumntype{A}{!{\vrule width 0.04em}}
\newcolumntype{B}{!{\vrule width 0.07em}}
\newcolumntype{C}{!{\vrule width 0.11em}}
%% More flexible list management, esp. for references
%% But also for specifying labels (i.e. custom order) on nested lists
\usepackage{enumitem}
%% hyperref driver does not need to be specified, it will be detected
%% Footnote marks in tcolorbox have broken linking under
%% hyperref, so it is necessary to turn off all linking
%% It *must* be given as a package option, not with \hypersetup
\usepackage[hyperfootnotes=false]{hyperref}
%% configure hyperref's  \url  to match listings' inline verbatim
\renewcommand\UrlFont{\small\ttfamily}
%% Hyperlinking active in electronic PDFs, all links solid and blue
\hypersetup{colorlinks=true,linkcolor=blue,citecolor=blue,filecolor=blue,urlcolor=blue}
\hypersetup{pdftitle={Math 584, Fall 2020 Syllabus}}
%% If you manually remove hyperref, leave in this next command
\providecommand\phantomsection{}
%% If tikz has been loaded, replace ampersand with \amp macro
%% Custom Preamble Entries, late (use latex.preamble.late)
%% Begin: Author-provided packages
%% (From  docinfo/latex-preamble/package  elements)
%% End: Author-provided packages
%% Begin: Author-provided macros
%% (From  docinfo/macros  element)
%% Plus three from MBX for XML characters

\newcommand{\lt}{<}
\newcommand{\gt}{>}
\newcommand{\amp}{&}
%% End: Author-provided macros
%% Title page information for article
\title{Math 584, Fall 2020 Syllabus}
\author{Zach Teitler\\
Department of Mathematics\\
Boise State University\\
Boise, Idaho, USA\\
\href{mailto:zteitler@boisestate.edu}{\nolinkurl{zteitler@boisestate.edu}}
}
\date{}
\begin{document}
%% Target for xref to top-level element is document start
\hypertarget{x:article:zteitler-2020C-584-syllabus}{}
\maketitle
\thispagestyle{empty}
%
%
\typeout{************************************************}
\typeout{Section 1 Course Information}
\typeout{************************************************}
%
\begin{sectionptx}{Course Information}{}{Course Information}{}{}{x:section:course-info}
\begin{introduction}{}%
%
\begin{description}
\item[{Course Number:}]MATH 584%
\item[{Course Title:}]Selected Topics: Computational Algebra%
\item[{Course Description:}]Introductory algebraic geometry based on computational methods in polynomial rings, particularly methods based on Gröbner bases.%
\end{description}
%
\end{introduction}%
%
%
\typeout{************************************************}
\typeout{Subsection 1.1 Instructor}
\typeout{************************************************}
%
\begin{subsectionptx}{Instructor}{}{Instructor}{}{}{x:subsection:instructor-info}
%
\begin{description}
\item[{Instructor:}]Zach Teitler%
\item[{Email:}]\href{mailto:zteitler@boisestate.edu}{\nolinkurl{zteitler@boisestate.edu}}%
\item[{Website:}]\url{https://sites.google.com/site/zteitler/home}%
\item[{Office:}]MB 233A%
\item[{Office Phone:}]208-426-1086%
\end{description}
%
\end{subsectionptx}
%
%
\typeout{************************************************}
\typeout{Subsection 1.2 Section}
\typeout{************************************************}
%
\begin{subsectionptx}{Section}{}{Section}{}{}{g:subsection:idm255884750656}
%
\begin{description}
\item[{Section Number:}]001%
\item[{Meeting Times:}]MoWeFr 12:00-1:15%
\item[{Meeting Remotely:}]We will meet remotely using Zoom. Zoom sessions may be recorded for students who are not able to attend.%
\item[{Zoom Room:}]\mono{984 1606 8308}%
\item[{Zoom Link:}]\url{https://boisestate.zoom.us/j/98416068308}%
\end{description}
%
\end{subsectionptx}
\end{sectionptx}
%
%
\typeout{************************************************}
\typeout{Section 2 Course Learning Outcomes}
\typeout{************************************************}
%
\begin{sectionptx}{Course Learning Outcomes}{}{Course Learning Outcomes}{}{}{g:section:idm255884449536}
By the end of this course, students will be able to:%
\begin{enumerate}
\item{}Demonstrate familiarity with introductory concepts, definitions, and theorems in algebraic geometry, including:%
\begin{itemize}[label=\textbullet]
\item{}Affine varieties and ideals%
\item{}Regular and rational maps%
\item{}Elimination and projection%
\item{}Hilbert's Nullstellensatz%
\end{itemize}
%
\item{}Demonstrate familiarity with introductory concepts, definitions, and theorems in computational algebra, including:%
\begin{itemize}[label=\textbullet]
\item{}Computations with multivariable polynomials, including division with remainder%
\item{}Monomial orderings%
\item{}Gröbner bases and Buchberger's algorithm%
\end{itemize}
%
\item{}Read and write rigorous proofs.%
\item{}Demonstrate good mathematical writing skills and style.%
\item{}Demonstrate familiarity with research and conjectures in algebraic geometry and computational algebra.%
\end{enumerate}
%
\end{sectionptx}
%
%
\typeout{************************************************}
\typeout{Section 3 Textbook}
\typeout{************************************************}
%
\begin{sectionptx}{Textbook}{}{Textbook}{}{}{g:section:idm255884662208}
%
%
\typeout{************************************************}
\typeout{Subsection 3.1 Required Textbook}
\typeout{************************************************}
%
\begin{subsectionptx}{Required Textbook}{}{Required Textbook}{}{}{g:subsection:idm255884692832}
%
\begin{enumerate}
\item{}Brendan Hassett, \pubtitle{Introduction to Algebraic Geometry}. Cambridge University Press, 2007.%
\begin{description}
\item[{ebook:}] 978-0-511-28529-5\item[{hardback:}] 978-0-521-87094-8\item[{paperback:}] 978-0-521-69141-3\end{description}
%
\end{enumerate}
%
\par
We will cover material from the chapters of the textbook listed above the dividing line in the following table; additional material from chapters listed after the dividing line may be covered if time permits: \begin{tableptx}{\textbf{}}{g:table:idm255883059840}{}%
\centering
{\tabularfont%
\begin{tabular}{ll}\hrulethick
Chapter&Title\tabularnewline\hrulethin
1&Guiding Problems\tabularnewline[0pt]
2&Division Algorithm and Gröbner Bases\tabularnewline[0pt]
3&Affine Varieties\tabularnewline[0pt]
4&Elimination\tabularnewline[0pt]
6&Irreducible Varieties\tabularnewline[0pt]
7&Nullstellensatz\tabularnewline\hrulethin
5&Resultants\tabularnewline[0pt]
8&Primary Decomposition\tabularnewline[0pt]
9&Projective Geometry\tabularnewline\hrulethick
\end{tabular}
}%
\end{tableptx}%
%
\end{subsectionptx}
%
%
\typeout{************************************************}
\typeout{Subsection 3.2 Recommended Optional Textbooks}
\typeout{************************************************}
%
\begin{subsectionptx}{Recommended Optional Textbooks}{}{Recommended Optional Textbooks}{}{}{g:subsection:idm255883147312}
%
\begin{enumerate}
\item{}Mateusz Michałek and Bernd Sturmfels, \pubtitle{Invitation to Nonlinear Algebra}. \url{https://personal-homepages.mis.mpg.de/michalek/NonLinearAlgebra.pdf} (pdf).%
\item{}Cox, Little, O'Shea, \pubtitle{Ideals, Varieties, and Algorithms}%
\end{enumerate}
%
\end{subsectionptx}
\end{sectionptx}
%
%
\typeout{************************************************}
\typeout{Section 4 Grading}
\typeout{************************************************}
%
\begin{sectionptx}{Grading}{}{Grading}{}{}{g:section:idm255884735872}
%
%
\typeout{************************************************}
\typeout{Subsection 4.1 Components of course grade}
\typeout{************************************************}
%
\begin{subsectionptx}{Components of course grade}{}{Components of course grade}{}{}{g:subsection:idm255884756496}
Graded student work will consist of a short paper and a term paper, midterm and final exams, and written homework. Course grades will be based primarily on term papers and exams, and to a lesser extent on homework.%
\end{subsectionptx}
%
%
\typeout{************************************************}
\typeout{Subsection 4.2 Papers}
\typeout{************************************************}
%
\begin{subsectionptx}{Papers}{}{Papers}{}{}{g:subsection:idm255883312256}
Each student will write two papers for this class. Each paper is intended to provide:%
\begin{itemize}[label=\textbullet]
\item{}an opportunity to explore the mathematical research literature and seek a personally appealing research topic%
\item{}an authentic experience of working with the published mathematical literature%
\item{}an authentic experience of writing a substantial mathematical text, well beyond typical homework solutions%
\item{}preparation for writing even an larger text, such as a thesis%
\item{}an authentic experience of receiving feedback on a draft manuscript and responding to that feedback through multiple rounds of revision%
\item{}an authentic experience of writing mathematical exposition at a high level%
\end{itemize}
%
\par
The details of the two papers are as follows.%
\begin{enumerate}
\item{}A short paper about an open question, unsolved problem, or currently studied research area within algebra, algebraic geometry, or computational algebra, or an application of one of those areas.%
\item{}A term paper about a topic of the student's choice within algebra, algebraic geometry, or computational algebra, or an application of one of those areas (but emphasizing the algebraic aspect).%
\end{enumerate}
%
\par
Lengths and deadlines of the papers are as follows:%
\begin{enumerate}
\item{}The short paper will be 2–3 pages long, with the following deadlines:%
\begin{itemize}[label=\textbullet]
\item{}A topic and source are due by the end of week 3.%
\item{}A first draft is due by the end of week 4.%
\item{}The short paper is due by the end of week 5.%
\end{itemize}
%
\item{}The term paper will be 6–15 pages long (it can be shorter or longer if needed; please discuss with me), with the following deadlines:%
\begin{itemize}[label=\textbullet]
\item{}A topic (1–2 pages) is due by the end of week 7.%
\item{}An outline plus one section are due by the end of week 10.%
\item{}A first draft is due by the end of week 13.%
\item{}The term paper is due by the end of week 15.%
\end{itemize}
%
\end{enumerate}
%
\par
For both papers, highly recommended sources are the recommended additional texts listed above. They contain numerous examples, applications of algebraic geometry, and links to other books and articles. If you find a topic that interests you in one of those texts, you can use it as a starting point for your paper.%
\par
Additional recommended sources include:%
\begin{itemize}[label=\textbullet]
\item{}\href{https://www.maa.org/programs-and-communities/member-communities/maa-awards/writing-awards}{MAA Writing Awards}%
\item{}\href{https://www-jstor-org.libproxy.boisestate.edu/journal/amermathmont}{American Mathematical Monthly}%
\item{}\href{https://www-jstor-org.libproxy.boisestate.edu/journal/collmathj}{College Mathematics Journal}%
\item{}\href{https://www-jstor-org.libproxy.boisestate.edu/journal/mathmaga}{Mathematics Magazine}%
\item{}\href{https://www-jstor-org.libproxy.boisestate.edu/journal/mathhorizons}{Math Horizons}%
\item{}\href{https://catalog.boisestate.edu/vwebv/holdingsInfo?bibId=358134}{What's Happening in the Mathematical Sciences}%
\end{itemize}
%
\par
You should use \terminology{high-quality published sources}. For simplicity, this means the recommended sources listed above, or any publication listed in \href{https://mathscinet.ams.org}{MathSciNet}. If you have questions or wish to use other sources, talk to me.%
\par
Papers for this class may not be about your own research.%
\end{subsectionptx}
%
%
\typeout{************************************************}
\typeout{Subsection 4.3 Midterm and final exams}
\typeout{************************************************}
%
\begin{subsectionptx}{Midterm and final exams}{}{Midterm and final exams}{}{}{g:subsection:idm255884658336}
A midterm exam in the 8th week of the semester, and the final exam, will be individual oral exams. The exams are intended to gauge mastery of course material, but also reflection on the course and its place in a larger context.%
\end{subsectionptx}
%
%
\typeout{************************************************}
\typeout{Subsection 4.4 Written homework}
\typeout{************************************************}
%
\begin{subsectionptx}{Written homework}{}{Written homework}{}{}{g:subsection:idm255884677280}
\begin{introduction}{}%
You are encouraged to work collaboratively on homework but you must turn in your own solutions.%
\par
Homework will contribute to course grades, but it is intended primarily to give feedback and guidance to students throughout the semester.%
\end{introduction}%
%
%
\typeout{************************************************}
\typeout{Subsubsection 4.4.1 Turning in homework}
\typeout{************************************************}
%
\begin{subsubsectionptx}{Turning in homework}{}{Turning in homework}{}{}{g:subsubsection:idm255884702192}
Homework submissions and grading will be paperless. You will turn in your homework by uploading PDFs to BlackBoard. PDFs should have filenames in the following format: \begin{quote}%
\mono{584-Homework-}\textlangle{}number\textrangle{}\mono{-}\textlangle{}your last name\textrangle{}\mono{.pdf}\end{quote}
 For example: \mono{584-Homework-01-Teitler.pdf}.%
\end{subsubsectionptx}
%
%
\typeout{************************************************}
\typeout{Subsubsection 4.4.2 Homework formatting}
\typeout{************************************************}
%
\begin{subsubsectionptx}{Homework formatting}{}{Homework formatting}{}{}{g:subsubsection:idm255884746000}
Homework must be typed in \LaTeX{}. \LaTeX{} tutorials are available online, e.g., \url{https://www.latex-tutorial.com} and \url{https://www.gnu.org/software/teximpatient/}. You may wish to use a free online \LaTeX{} system such as \url{https://overleaf.com}. (Overleaf includes a \LaTeX{} tutorial.)%
\par
Use a new page (\mono{\textbackslash{}newpage}) for each problem. State which question you are answering (textbook section and exercise number) and the actual question. Then, start your answer in a new paragraph. Use environments such as \mono{proof} and \mono{theorem} (via \mono{\textbackslash{}begin\{proof\}}\textellipsis{}\mono{\textbackslash{}end\{proof\}}) to organize your work and display it clearly.%
\par
For legibility, use the \mono{12pt} option (\mono{\textbackslash{}documentclass[12pt]\{amsart\}}) and \mono{\textbackslash{}linespread\{2.4\}}. If you use figures, I recommend learning to use \mono{TikZ} to generate high-quality figures within \LaTeX{}. Alternatively you may use figures\slash{}plots generated in other programs such as Sage, Mathematica, Maple, or Inkscape, saved to PDF, and included in your document with commands like \mono{\textbackslash{}includegraphics}. It's also fine to include hand-drawn figures that you scanned or photoed.%
\end{subsubsectionptx}
\end{subsectionptx}
\end{sectionptx}
%
%
\typeout{************************************************}
\typeout{Section 5 Help}
\typeout{************************************************}
%
\begin{sectionptx}{Help}{}{Help}{}{}{g:section:idm255883222032}
%
%
\typeout{************************************************}
\typeout{Subsection 5.1 Allowed resources}
\typeout{************************************************}
%
\begin{subsectionptx}{Allowed resources}{}{Allowed resources}{}{}{g:subsection:idm255883225568}
In this 500 level class, I expect you to use the full array of resources at your disposal to learn the material ``by any means possible.'' For homework assignments, you are encouraged (in fact, expected) to collaborate with your classmates and to ask me questions. Other resources (books, online sources, people outside the class) are highly recommended for clarifying class topics or for enrichment (but \emph{not} for getting solutions to problems).%
\par
You are allowed to use things that you learn from a book, online source, or person outside the class that \emph{help you and your classmates to find your own solution} for a problem. However, if you read a full solution, so that there's little or nothing left for you and your classmates to figure out, then you \emph{may not turn in that solution for credit}.%
\par
The \href{https://math.stackexchange.com}{Mathematics Stack Exchange} is a very useful question-and-answer site for undergraduate\slash{}graduate level mathematics. You are welcome to browse the Mathematics Stack Exchange and even post questions there. Hopefully it will help you learn and understand the material! However please remember that if you use that site to get a solution to a problem, then you can't turn that solution in for credit. Other helpful websites for basic information include Wikipedia and \href{http://mathworld.wolfram.com}{MathWorld}.%
\end{subsectionptx}
%
%
\typeout{************************************************}
\typeout{Subsection 5.2 A note on collaboration}
\typeout{************************************************}
%
\begin{subsectionptx}{A note on collaboration}{}{A note on collaboration}{}{}{g:subsection:idm255883460000}
Solving mathematical problems has three parts:%
\begin{description}
\item[{The discovery phase:}]This is the time you spent trying to figure out how to solve the problems, and it often takes most of the time. You are welcome and encouraged to collaborate with other students in this phase. Collaboration is a healthy practice, and this is how mathematics is done in real life.%
\par
This phase starts with working to understand what a problem or question is asking for. That might include reviewing material from previous textbook sections.%
\item[{The write-up phase:}]This consists of writing your solutions once you have an idea of how the problem can be solved. You should do this entirely by yourself. Be alone when you write your solutions. If you collaborate on this part, or you copy part of your solutions from somebody else, or you have notes written by somebody else in front of you when you write your solutions, you are hurting yourself by depriving yourself of an opportunity to learn and practice. %
\par
If you need help in the write-up phase, talk to me! I can help you.%
\item[{The editing phase:}]This consists of editing and revising your write-up for clarity, organization, and presentation. At this stage it can be very helpful to get feedback and suggestions from other students.%
\end{description}
%
\end{subsectionptx}
%
%
\typeout{************************************************}
\typeout{Subsection 5.3 Expectations of lectures}
\typeout{************************************************}
%
\begin{subsectionptx}{Expectations of lectures}{}{Expectations of lectures}{}{}{g:subsection:idm255898125824}
Please read ``\href{http://blogs.ams.org/matheducation/2015/02/10/}{Mathematics Professors and Mathematics Majors' Expectations of Lectures in Advanced Mathematics}''.%
\end{subsectionptx}
%
%
\typeout{************************************************}
\typeout{Subsection 5.4 How to read mathematics}
\typeout{************************************************}
%
\begin{subsectionptx}{How to read mathematics}{}{How to read mathematics}{}{}{g:subsection:idm255897979424}
Please read the following, especially the first:%
\begin{enumerate}
\item{}\url{https://www.lboro.ac.uk/media/media/schoolanddepartments/mathematics-education-centre/downloads/research/SE-booklet.pdf}%
\item{}\url{https://personal.utdallas.edu/\~zweck/SEbooklet.pdf}%
\item{}\url{http://web.stonehill.edu/compsci/History_Math/math-read.htm}%
\item{}\url{https://brownmath.com/stfa/read.htm}%
\end{enumerate}
%
\end{subsectionptx}
\end{sectionptx}
%
%
\typeout{************************************************}
\typeout{Section 6 Important Dates}
\typeout{************************************************}
%
\begin{sectionptx}{Important Dates}{}{Important Dates}{}{}{g:section:idm255884529776}
\begin{tableptx}{\textbf{}}{g:table:idm255884548288}{}%
\centering
{\tabularfont%
\begin{tabular}{lll}\hrulethick
Monday&8\slash{}24&First day of classes.\tabularnewline[0pt]
Friday&9\slash{}4&Last day to register\slash{}add or to drop without a W.\tabularnewline[0pt]
Monday&9\slash{}7&Labor Day. No classes.\tabularnewline[0pt]
Friday&9\slash{}11&Short paper topic proposal due.\tabularnewline[0pt]
Friday&9\slash{}18&Short paper first draft due.\tabularnewline[0pt]
Friday&9\slash{}25&Short paper final version due.\tabularnewline[0pt]
Friday&10\slash{}9&Term paper topic proposal due.\tabularnewline[0pt]
&10\slash{}12-16&Midterm exam.\tabularnewline[0pt]
Friday&10\slash{}30&Term paper outline plus one section due\tabularnewline[0pt]
Friday&10\slash{}30&Last day to drop with a W or completely withdraw.\tabularnewline[0pt]
Friday&11\slash{}20&Term paper first draft due.\tabularnewline[0pt]
&11\slash{}23-11\slash{}29&Thanksgiving Holiday. No classes.\tabularnewline[0pt]
Friday&12\slash{}11&Term paper final version due.\tabularnewline[0pt]
Friday&12\slash{}11&Last day of instruction for regular classes.\tabularnewline[0pt]
Monday&12\slash{}14&Final Exam (scheduled), 12:00-2:00.\tabularnewline[0pt]
Tuesday&12\slash{}22&Grades due. (You will be able to see your grade by this date.)\tabularnewline\hrulethick
\end{tabular}
}%
\end{tableptx}%
\end{sectionptx}
%
%
\typeout{************************************************}
\typeout{Section 7 Other}
\typeout{************************************************}
%
\begin{sectionptx}{Other}{}{Other}{}{}{g:section:idm255884549328}
%
\begin{description}
\item[{Respect for Diversity:}]Students from all backgrounds and with all perspectives are welcome in this course. It is my intent that all students be well served by this course, that students's learning needs be addressed both in and out of class, and that the diversity that students bring to this class be viewed as a resource, strength, and benefit. It is my intent to maintain a classroom atmosphere that is welcoming and respectful of diversity: gender, sexuality, disability, age, socioeconomic status, ethnicity, race, and culture. Your suggestions are encouraged and appreciated. Please let me know ways to improve the effectiveness of the course for you personally or for other students or student groups. %
\item[{ADA Policy Statement:}]Students with disabilities needing accommodations to fully participate in this class should contact the EAC. All accommodations must be approved through the EAC prior to being implemented. To learn more about the accommodation process, visit the EAC's website at \url{https://www.boisestate.edu/eac/new-students/}.%
\item[{Email:}]In accordance with \href{http://boisestate.edu/policy/policy-title-student-e-mail-communications/}{Boise State University Policy \#2280}, it is expected that you will receive and read emails sent to your \mono{boisestate.edu} email address.%
\item[{Communication:}]Additional information and updates may be announced in class, sent by email, and\slash{}or posted on BlackBoard (\url{http://blackboard.boisestate.edu/}).%
\item[{Academic Integrity:}]Getting answers to homework or exam problems from unauthorized sources is a very serious form of academic misconduct. For this class, \emph{all online sources} are unauthorized for this purpose. You are allowed to \emph{learn and increase your understanding} from online sources or other textbooks; you are \emph{not} allowed to use those sources to find answers to homework or exam problems.%
\item[{Behavioral Expectations:}]Every student has the right to a respectful learning environment. In order to provide this right to all students, students must take individual responsibility to conduct themselves in a mature and appropriate manner and will be held accountable for their behavior in accordance with \href{http://boisestate.edu/policy/student-affairs/maintaining-order/}{Boise State University Policy \#2050}.%
\end{description}
%
\end{sectionptx}
\end{document}